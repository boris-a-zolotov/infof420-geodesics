\documentclass[a4paper,11pt]{article}
\usepackage{subfiles}
\usepackage{modules/f420}

\title{Geodesics, Shortest Paths, and Non-Isomorphic Nets}
\author{{\scshape Potvin} Nicolas \and {\scshape Zolotov} Boris}
\date{ULB,\quad November 2020}

\begin{document} \maketitle

\section{Introduction}

Given a collection of 2D polygons, a \emph{gluing} or a \emph{net} describes a closed surface by specifying how to glue (a part of) each edge of these polygons onto (a part of) another edge. Alexandrov's uniqueness theorem~\cite{alex} states that any valid gluing that is homeomorphic to a sphere and that does not yield a total facial angle greater than $2\pi$ at any point, corresponds to the surface of a unique convex 3D polyhedron (doubly covered convex polygons are also regarded as polyhedra). Note that the original polygonal pieces might need to be folded to obtain this 3D surface.

Unfortunately, the proof of Alexandrov's theorem is highly non-constructive. The only known approximation algorithm to find the vertices of this polyhedron~\cite{kpd09-approx} has a (pseudopolynomial) running time really large in $n$, where $n$ is the total complexity of the gluing.

There is no known exact algorithm for reconstructing the 3D polyhedron, and in fact the coordinates of the vertices of the polyhedron might not even be expressible as a closed formula~\cite{bannister2014galois}.

Enumerating all possible valid gluings is also not an easy task, as the number of gluings can be exponential even for a single polygon~\cite{DDLO02}. However one valid gluing can be found in polynomial time using dynamic programming~\cite{DO07,lo96-dynprog}. Complete enumerations of gluings and the resulting polyhedra are only known for very specific cases such as the Latin cross~\cite{ddlop99} and a single regular convex polygon~\cite{DO07}.

The special case when the polygons to be glued together are all identical regular $k$-gons, and the gluing is \emph{edge-to-edge} was studied recently for $k \ge 6$~\cite{kl17-hex} and $k=5$~\cite{alz-penta}. The aim of this project is to study the case of $k=4$.

\subfile{NicolasPart.tex}

\section{Boris' part: enumeration of gluings, isomorphism check}

Estimate of the number of vertices, enumeration of gluings was done in~\cite{kl17-hex,alz-penta}.

%\section{Interface between the two parts}
\subfile{APISpecs.tex}

\bibliography{boris-bac}{}
\bibliographystyle{plain}

\end{document}
