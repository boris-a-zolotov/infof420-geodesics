\documentclass[a4paper,USenglish,cleveref, autoref, thm-restate]{socg-lipics-v2019}

\usepackage{algorithm,algpseudocode}
\usepackage{tikz}
\usetikzlibrary{arrows,backgrounds,patterns,matrix,
	shapes,fit,calc,shadows,plotmarks,snakes}

\newcommand{\baserect}{
	\foreach \x in {0,...,6} {\draw[gray,opacity=0.8] (\x,0) -- (\x,5);}
	\foreach \y in {0,...,5} {\draw[gray,opacity=0.8] (0,\y) -- (6,\y);}
	\draw[thick] (0,0) -- (6,0) -- (6,5) -- (0,5) -- cycle;
}

\definecolor{bluetri}{RGB}{45,155,190}

\newcommand{\midnode}{
	node[midway,circle,fill=white,fill opacity=0.92,inner sep=0pt,minimum size=2.1ex]
}

\bibliographystyle{plainurl}

\title{Enumerating All Convex Polyhedra Glued from Squares in Polynomial Time}

\titlerunning{Convex Polyhedra Glued from Squares}

\author{Stefan Langerman}{Faculté des Sciences, Université Libre de Bruxelles}{stefan.langerman@ulb.ac.be}{ }{Whatever grant}

\author{Nicolas Potvin}{Faculté des Sciences, Université Libre de Bruxelles}{potvinnicolas2@gmail.com}{ }{Whatever grant}

\author{Boris Zolotov}{Department of Mathematics and Computer Sciences, St. Petersburg State University}{boris.a.zolotov@yandex.com}{ }{is supported in part by the Foundation for the Advancement of Theoretical Physics and Mathematics ``BASIS'' and in part by ``Native towns'', a social investment program of PJSC ``Gazprom Neft''.}

\authorrunning{S. Langerman and N. Potvin and B. Zolotov}

\Copyright{Stefan Langerman and Nicolas Potvin and Boris Zolotov}

\keywords{polyhedral metrics, alexandrov theorem, squares, edge-to-edge gluings}

\acknowledgements{B. Z. wants to thank the international academic mobility \& students exchange program of SPBU for offering him an exchange in ULB that allowed for cooperation needed to complete this project.}

\ccsdesc{\#10010061 Computational geometry}

% \relatedversion{0}

\begin{document}

\maketitle

\begin{abstract}We present a polynomial-time algorithm that enumerates and classifies all edge-to-edge gluings of squares that correspond to convex polyhedra. We show that the number of such gluings of $n$ squares is polynomial in $n$. Our technique can be applied in several similar settings, including gluings of regular hexagons and triangles.\end{abstract}

\section{Introduction}

Given a collection of 2D polygons, a \emph{gluing} or a \emph{net} describes a closed surface by specifying how to glue each edge of these polygons onto another edge. Alexandrov's uniqueness theorem~\cite{alex} states that any valid gluing that is homeomorphic to a sphere and that does not yield a total facial angle greater than $2\pi$ at any point, corresponds to the surface of a unique convex 3D polyhedron (doubly covered convex polygons are also polyhedra).

There is no known exact algorithm for reconstructing the 3D polyhedron~\cite{bannister2014galois,kpd09-approx}. Enumerating all possible valid gluings is also not an easy task, as the number of gluings can be exponential even for a single polygon~\cite{DDLO02}. Complete enumerations of gluings and the resulting polyhedra are only known for very specific cases such as the Latin cross~\cite{ddlop99}, a single regular convex polygon~\cite{DO07}, and a collection of regular pentagons~\cite{alz-penta}.

The case when the polygons to be glued together are all identical regular $k$-gons, and the gluing is \emph{edge-to-edge} was studied recently for $k \ge 6$~\cite{kl17-hex}. The aim of this paper is to study the case of $k=4$: namely, to {\it enumerate} all valid gluings of squares and {\it classify} them up to isomorphism.

\section{Chen—Han algorithm for gluings of squares}

In~\cite{DO07} it is shown that polyhedra are isomorphic if the lengths of shortest geodesic paths between their vertices of nonzero curvature coincide. Thus, the problem of finding out if two gluings are isomorphic can be reduced to finding out the geodesic distances between vertices of a gluing. Algorithm we are using for this is the Chen—Han algorithm~\cite{chen-han}.

The idea of the algorithm is to project a cone of all possible paths from the source onto the surface of the gluing. This algorithm runs in $O(n^2)$ time. To apply it for arbitrary edge-to-edge gluings of squares, it has to be proven that the running time is preserved. To do this, we show that there is at most one segment of a geodesic shortest path for each pair of adjacent sides of a square with endpoints on these sides, and at most one segment whose endpoints lie on opposite sides of a square. This implies the following Theorem.

\begin{theorem} \label{thm:shortestSquare}
	If $T$ is a square of the gluing and $\pi$ is a geodesic shortest path between two vertices of the gluing then the intersection between $\pi$ and $T$ is of at most 5 segments.
\end{theorem}

%	\begin{corollary} \label{cor:chruntime}
%	For an arbitrary gluing $\Gamma$ of $n$ squares and a vertex $v$
%	of this gluing, Chen—Han algorithm for $\Gamma$ with source $v$
%	runs in $O(n^2)$ time.
%	\end{corollary}

\section{Bounds on the number of egde-to-edge gluings of squares}

In this section, we prove that the number of edge-to-edge gluings of $n$ squares is polynomial in $n$. This result allows to develop a polynomial algorithm to list all the gluings.

\begin{theorem} \label{thm:n36}
	There are $O \left( n^{36} \right)$ edge-to-edge gluings of at most $n$ squares that correspond to convex polyhedra.
\end{theorem}

\begin{proof}
	Triangulate the polyhedron corresponding to the net and draw its faces on the square grid. An edge shared by two faces must have the same lengths of $x$- and $y$-projections on the drawings of these faces, see Figures~\ref{fig:emA},~\ref{fig:emB}. Then count the number of sets of triangles satisfying this restriction and taking up at most $n$ squares. To do so, choose the projections (that do not exceed $n$ in length) for each of at most 18 edges. Note that there are at most two ways to place each edge that preserve convexity of the face, see Figure~\ref{fig:emC}.
\end{proof}

\begin{figure}[h] \centering
\begin{subfigure}[t]{3.6cm} \centering
\tikz[scale=0.45]{
	\foreach \i in {-1,...,6} {
		\draw[gray] (-1,\i) -- (6,\i) (\i,-1) -- (\i,6);
	}
	\draw[very thick,red] (4,5) -- (5,2);
	\draw[thick] (5,2) -- ++(-2,-2) -- ++(-3,0) --
		++(0,2) -- ++(1,3) -- ++(3,0);
} \caption{} \label{fig:emA} \end{subfigure} \hspace{0.6cm}
\begin{subfigure}[t]{3.6cm} \centering
\tikz[scale=0.45]{
	\foreach \i in {-1,...,6} {
		\draw[gray] (-1,\i) -- (6,\i) (\i,-1) -- (\i,6);
	}
	\draw[very thick,red] (0,4) -- (3,5);
	\draw[thick] (3,5) -- ++(2,-2) -- ++(0,-3) --
		++(-3,0) -- ++(-2,2) -- ++(0,2);
} \caption{} \label{fig:emB} \end{subfigure} \hspace{0.8cm}
\begin{subfigure}[t]{3.2cm} \centering
\tikz[scale=0.45]{
	\foreach \i in {0,...,6} {
		\draw[gray] (0,\i) -- (6,\i) (\i,0) -- (\i,6);
	}
	\draw[thick] (1,0)--(2,3);
	\draw[thick,red,dashed] (4,6)--(2,3)--(5,1);
} \caption{} \label{fig:emC} \end{subfigure}

\caption{(a), (b) Highlighted edge has the same lengths of projections on
	the drawings of two faces. (c) Two ways to place an edge with given
	projections that preserve convexity of the face.}
\end{figure}


\begin{theorem} \label{thm:n52}
	There are $\Omega \left( n^{\frac52} \right)$ edge-to-edge gluings of at most $n$ squares that satisfy Alexandrov's conditions.
\end{theorem}

\begin{proof}
	To prove the theorem, we construct a series of such gluings. These gluings correspond to doubly-covered polygons, the polygons being obtained by cutting edges of a rectangle with sides no longer than $\frac{\sqrt{n}}{2}$, see Figure~\ref{fig:cutEx}.
\end{proof}

\begin{figure}[h] \centering
\begin{subfigure}[t]{3.2cm} \centering
\tikz[scale=0.45]{ % (3,-0.7) node[text height=3ex]{(a)}
	\baserect
	\filldraw[fill=bluetri,thick,fill opacity=0.55] (6,0) -- (4,0) -- (6,2) -- (3,5) -- (6,5) -- cycle;
	\filldraw[fill=bluetri,thick,fill opacity=0.55] (0,0) -- (1,0) -- (0,1) -- cycle;
	\filldraw[fill=bluetri,thick,fill opacity=0.55] (0,3) -- (2,5) -- (0,5) -- cycle;
	\fill (6,2) circle[radius = 1.6mm]
		(0,3) circle[radius = 1.6mm]
		(0,1) circle[radius = 1.6mm];
} \caption{} \label{fig:cutExA} \end{subfigure} \hspace{0.6cm}
\begin{subfigure}[t]{3.2cm} \centering
\tikz[scale=0.45]{
	\baserect
	\filldraw[fill=bluetri,thick,fill opacity=0.55] (0,3) -- (0,5) -- (6,5) -- (6,1) -- (2,5) -- cycle;
	\draw[thick,red,dashed] (0,2) -- (3,5);
	\fill (6,1) circle[radius = 1.6mm]
		(0,2) circle[radius = 1.6mm]
		(0,3) circle[radius = 1.6mm]
		(0,0) circle[radius = 1.6mm]
		(6,0) circle[radius = 1.6mm];
} \caption{} \label{fig:cutExB} \end{subfigure} \hspace{0.6cm}
\begin{subfigure}[t]{3.2cm} \centering
\tikz[scale=0.45]{
	\baserect
	\filldraw[fill=bluetri,thick,fill opacity=0.55] (0,0) -- (4,4) -- (6,2) -- (6,5) -- (0,5) -- cycle;
	\draw[thick,dashed] (0,4) -- (6,4);
	\fill (0,0) circle[radius = 1.6mm]
		(6,0) circle[radius = 1.6mm]
		(6,2) circle[radius = 1.6mm];
} \caption{} \label{fig:cutExC} \end{subfigure}

\caption{(a) An example of a polygon produced by cutting angles of a rectangle. (b) Some ways to cut angles do not produce valid polygons. (c) A polygon can be obtained by cutting angles of several different rectangles.} \label{fig:cutEx}
\end{figure}


We implemented an algorithm that enumerates all the gluings of at most $n$ squares for a given graph structure of a convex polyhedron. It showed that one gluing can admit several ways to cut itself into flat polygons, see Figure~\ref{fig:twonets}. Thus it can appear in the list several times.

\begin{figure}[h] \centering
      \begin{subfigure}[b]{0.44\textwidth} \centering
            \tikz[scale=0.56]{\fill[white] (-0.4,-0.4) rectangle (4.4,4.4);
                  \foreach \i in {0,...,3} {\draw[gray,opacity=0.6] (\i,0) -- (\i,4);}
                  \foreach \i in {0,...,4} {\draw[gray,opacity=0.6] (0,\i) -- (3,\i);}
                  \draw[thick] (2,3) -- (0,0) \midnode{0} --
                        (1,1) \midnode{1} -- (3,4) \midnode{2} --
                        cycle \midnode{3};}
            \quad
            \tikz[scale=0.56]{\fill[white] (-0.4,-0.4) rectangle (4.4,4.4);
                  \foreach \i in {0,...,3} {\draw[gray,opacity=0.6] (\i,0) -- (\i,4);}
                  \foreach \i in {0,...,4} {\draw[gray,opacity=0.6] (0,\i) -- (3,\i);}
                  \draw[thick] (3,3) -- (2,4) \midnode{3} --
                        (0,1) \midnode{2} -- (1,0) \midnode{1} --
                        cycle \midnode{0};}
      \caption{} \end{subfigure} \qquad
      \begin{subfigure}[b]{0.44\textwidth} \centering
            \tikz[scale=0.56]{\fill[white] (-0.4,-0.4) rectangle (4.4,4.4);
                  \foreach \i in {0,...,3} {\draw[gray,opacity=0.6] (\i,0) -- (\i,4);}
                  \foreach \i in {0,...,4} {\draw[gray,opacity=0.6] (0,\i) -- (3,\i);}
                  \draw[dashed] (2,3)--(1,1.5) (2,2.5)--(1,1);
                  \draw[thick] (1,4) -- (1,1) \midnode{0} --
                        (2,0) \midnode{1} -- (2,3) \midnode{2} --
                        cycle \midnode{3};}
            \quad
            \tikz[scale=0.56]{\fill[white] (-0.4,-0.4) rectangle (4.4,4.4);
                  \foreach \i in {0,...,3} {\draw[gray,opacity=0.6] (\i,0) -- (\i,4);}
                  \foreach \i in {0,...,4} {\draw[gray,opacity=0.6] (0,\i) -- (3,\i);}
                  \draw[dashed] (2,4)--(1,2.5) (2,1.5)--(1,0);
                  \draw[thick] (2,4) -- (2,1) \midnode{0} --
                        (1,0) \midnode{1} -- (1,3) \midnode{2} --
                        cycle \midnode{3};}
      \caption{} \end{subfigure}
      \caption{Doubly covered parallelogram can be cut into two flat quadrilaterals in two ways, the latter consisting of its faces}
      \label{fig:twonets}
\end{figure}


\section{Algorithm to classify edge-to-edge gluings of squares}

The algorithm consists of the following steps:

\begin{enumerate}
	\item Generate the list of all edge-to-edge gluings of at most $n$ squares,
	denote it $L(n)$. Due to Theorem~\ref{thm:n36}, this step takes polynomial time.
	\item For each gluing in $L(n)$, generate matrix of pairwise distances
	between its vertices. Due to Theorem~\ref{thm:shortestSquare},
	this step takes $O(n^3)$ time per gluing.
	\item Unicalize the list of matrices up to homothety and permutation of rows and columns, leave only corresponding elements of $L(n)$. Since the matrices are of at most 8 rows and 8 columns, the list takes polynomial time to remove duplicates.
\end{enumerate}

The output of this algorithm is the list of all non-isomorphic edge-to-edge gluings of at most $n$ squares.

\section{Discussion}

The cornerstone of the technique we have been using is the possibility to draw a face of a polyhedron glued from squares on a planar grid. It allows us to estimate the number of valid gluings. The same technique can seemingly be applied for the cases of regular hexagons and triangles, since these polygons also tile the plane.

\bibliography{boris-bac}

\end{document}
