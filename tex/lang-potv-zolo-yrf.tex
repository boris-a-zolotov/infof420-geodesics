\documentclass[a4paper,USenglish,cleveref, autoref, thm-restate]{socg-lipics-v2019}

\usepackage{algorithm,algpseudocode}
\usepackage{tikz}
\usetikzlibrary{arrows,backgrounds,patterns,matrix,
	shapes,fit,calc,shadows,plotmarks,snakes}

\newcommand{\baserect}{
	\foreach \x in {0,...,6} {\draw[gray,opacity=0.8] (\x,0) -- (\x,5);}
	\foreach \y in {0,...,5} {\draw[gray,opacity=0.8] (0,\y) -- (6,\y);}
	\draw[thick] (0,0) -- (6,0) -- (6,5) -- (0,5) -- cycle;
}

\definecolor{bluetri}{RGB}{45,155,190}

\newcommand{\midnode}{
	node[midway,circle,fill=white,fill opacity=0.92,inner sep=0pt,minimum size=2.1ex]
}

\bibliographystyle{plainurl}

\title{Enumerating All Convex Polyhedra Glued from Squares in Polynomial Time}

\titlerunning{Convex Polyhedra Glued from Squares}

\author{Stefan Langerman}{Faculté des Sciences, Université Libre de Bruxelles}{stefan.langerman@ulb.ac.be}{ }{Whatever grant}

\author{Nicolas Potvin}{Faculté des Sciences, Université Libre de Bruxelles}{potvinnicolas2@gmail.com}{ }{Whatever grant}

\author{Boris Zolotov}{Department of Mathematics and Computer Sciences, St. Petersburg State University}{boris.a.zolotov@yandex.com}{ }{Whatever grant}

\authorrunning{S. Langerman and N. Potvin and B. Zolotov}

\Copyright{Stefan Langerman and Nicolas Potvin and Boris Zolotov}

\keywords{polyhedral metrics, alexandrov theorem, squares, edge-to-edge gluings}

\acknowledgements{I want to thank \dots}

\ccsdesc{\#10010061 Computational geometry}

% \relatedversion{0}

\begin{document}

\maketitle

\begin{abstract}We present a polynomial-time algorithm that allows to enumerate and classify all edge-to-edge gluings of squares that correspond to convex polyhedra. We show that the number of such gluings of $n$ squares is polynomial in $n$. The methods we use to acheve this can be applied in several similar settings, including gluings of regular hexagons and triangles.\end{abstract}

\section{Introduction}

Given a collection of 2D polygons, a \emph{gluing} or a \emph{net} describes a closed surface by specifying how to glue (a part of) each edge of these polygons onto (a part of) another edge. Alexandrov's uniqueness theorem~\cite{alex} states that any valid gluing that is homeomorphic to a sphere and that does not yield a total facial angle greater than $2\pi$ at any point, corresponds to the surface of a unique convex 3D polyhedron (doubly covered convex polygons are also regarded as polyhedra). Note that the original polygonal pieces might need to be folded to obtain this 3D surface.

There is no known exact algorithm for reconstructing the 3D polyhedron~\cite{kpd09-approx,bannister2014galois}. In fact the coordinates of the vertices of the polyhedron might not even be expressible as a closed formula.

Enumerating all possible valid gluings is also not an easy task, as the number of gluings can be exponential even for a single polygon~\cite{DDLO02}. However one valid gluing can be found in polynomial time using dynamic programming~\cite{DO07,lo96-dynprog}. Complete enumerations of gluings and the resulting polyhedra are only known for very specific cases such as the Latin cross~\cite{ddlop99} and a single regular convex polygon~\cite{DO07}.

The special case when the polygons to be glued together are all identical regular $k$-gons, and the gluing is \emph{edge-to-edge} was studied recently for $k \ge 6$~\cite{kl17-hex} and $k=5$~\cite{alz-penta}. The aim of this paper is to study the case of $k=4$: namely, to {\it enumerate} all valid gluings of squares and {\it classify} them up to isomorphism.

\section{Adaptation of Chen—Han algorithm for arbitrary edge-to-edge gluings of squares}

In this section, we show the runtime of Chen—Han algorithm, whose input is an edge-to-edge gluing of squares, and the squares need not to be faces of the resulting polyhedron.

\begin{theorem} \label{thm:shortestSquare}
	If $T$ is a square of the gluing and $\pi$ is a geodesic shortest path between two vertices $p_1$, $p_2$ of the gluing then the intersection between $\pi$ and $T$ consists of at most 5 segments.
\end{theorem}

\begin{proof}
	To prove this theorem, we show that there is at most one segment for each pair of adjacent sides of $T$ with endpoints on these sides, and at most one segment whose endpoints lie on opposite sides of $T$.
\end{proof}

\begin{corollary} \label{cor:chruntime}
	For an arbitrary gluing $\Gamma$ of $n$ squares and a vertex $v$ of this gluing, Chen—Han algorithm for $\Gamma$ and $v$ runs in $O(n^2)$ time.
\end{corollary}

\section{Bounds on the number of egde-to-edge gluings of squares}

In this section, we prove that the number of edge-to-edge gluings of $n$ squares is polynomial in $n$. This result allows to develop a polynomial algorithm to list all the valid gluings.

\begin{theorem} \label{thm:n36}
	There are $O \left( n^{36} \right)$ edge-to-edge gluings of at most $n$ squares that satisfy Alexandrov's conditions.
\end{theorem}

\begin{proof} To prove the theorem, we triangulate the polyhedron corresponding to the net and draw its faces on the square grid. We note that an edge shared by two faces must have the same lengths of $x$- and $y$-projections on the drawings of these faces, see Figure~\ref{fig:edgesMeeting}. Then we count the number of sets of triangles satisfying this restriction and taking up at most $n$ squares.

\begin{figure}[h] \centering
\tikz[scale=0.45]{
	\foreach \i in {-1,...,6} {
		\draw[gray] (-1,\i) -- (6,\i) (\i,-1) -- (\i,6);
	}
	\draw[very thick,red] (4,5) -- (5,2);
	\draw[thick] (5,2) -- ++(-2,-2) -- ++(-3,0) --
		++(0,2) -- ++(1,3) -- ++(3,0);
}\hspace{1.2cm}
\tikz[scale=0.45]{
	\foreach \i in {-1,...,6} {
		\draw[gray] (-1,\i) -- (6,\i) (\i,-1) -- (\i,6);
	}
	\draw[very thick,red] (0,4) -- (3,5);
	\draw[thick] (3,5) -- ++(2,-2) -- ++(0,-3) --
		++(-3,0) -- ++(-2,2) -- ++(0,2);
}
\caption{Highlighted edge has the same lengths of projections on the drawings of two faces}
\label{fig:edgesMeeting}
\end{figure}

To do so, we choose $x$- and $y$-projections for each of at most 18 edges and note that there is at most two ways to place each edge such that the convexity of the face is preserved, those differ by $\frac{\pi}{2}$, see Figure~\ref{fig:twoWays}. \end{proof}

\begin{figure}[h] \centering
\tikz[scale=0.45]{
	\foreach \i in {0,...,6} {
		\draw[gray] (0,\i) -- (6,\i) (\i,0) -- (\i,6);
	}
	\draw[thick] (1,0)--(2,3);
	\draw[thick,red,dashed] (4,6)--(2,3)--(5,1);
}
\caption{There are two ways to place each edge preserving convexity of the face}
\label{fig:twoWays}
\end{figure}

\begin{theorem} \label{thm:n52}
	There are $\Omega \left( n^{\frac52} \right)$ edge-to-edge gluings of at most $n$ squares that satisfy Alexandrov's conditions.
\end{theorem}

\begin{proof} To prove the theorem, we construct a series of such gluings. These gluings correspond to doubly-covered polygons, the polygons being obtained by cutting edges of a rectangle with sides no longer than $\frac{\sqrt{n}}{2}$, see Figure~\ref{fig:cutExA}.

\begin{figure}[h] \centering
\begin{subfigure}[t]{3.2cm} \centering
\tikz[scale=0.45]{ % (3,-0.7) node[text height=3ex]{(a)}
	\baserect
	\filldraw[fill=bluetri,thick,fill opacity=0.55] (6,0) -- (4,0) -- (6,2) -- (3,5) -- (6,5) -- cycle;
	\filldraw[fill=bluetri,thick,fill opacity=0.55] (0,0) -- (1,0) -- (0,1) -- cycle;
	\filldraw[fill=bluetri,thick,fill opacity=0.55] (0,3) -- (2,5) -- (0,5) -- cycle;
	\fill (6,2) circle[radius = 1.6mm]
		(0,3) circle[radius = 1.6mm]
		(0,1) circle[radius = 1.6mm];
} \caption{} \label{fig:cutExA} \end{subfigure} \hspace{0.6cm}
\begin{subfigure}[t]{3.2cm} \centering
\tikz[scale=0.45]{
	\baserect
	\filldraw[fill=bluetri,thick,fill opacity=0.55] (0,3) -- (0,5) -- (6,5) -- (6,1) -- (2,5) -- cycle;
	\draw[thick,red,dashed] (0,2) -- (3,5);
	\fill (6,1) circle[radius = 1.6mm]
		(0,2) circle[radius = 1.6mm]
		(0,3) circle[radius = 1.6mm]
		(0,0) circle[radius = 1.6mm]
		(6,0) circle[radius = 1.6mm];
} \caption{} \label{fig:cutExB} \end{subfigure} \hspace{0.6cm}
\begin{subfigure}[t]{3.2cm} \centering
\tikz[scale=0.45]{
	\baserect
	\filldraw[fill=bluetri,thick,fill opacity=0.55] (0,0) -- (4,4) -- (6,2) -- (6,5) -- (0,5) -- cycle;
	\draw[thick,dashed] (0,4) -- (6,4);
	\fill (0,0) circle[radius = 1.6mm]
		(6,0) circle[radius = 1.6mm]
		(6,2) circle[radius = 1.6mm];
} \caption{} \label{fig:cutExC} \end{subfigure}

\caption{(a) An example of a polygon produced by cutting angles of a rectangle. (b) Some ways to cut angles do not produce valid polygons. (c) A polygon can be obtained by cutting angles of several different rectangles.} \label{fig:cutEx}
\end{figure}


If the length of the shorter side of the rectangle is equal to $b$, then there is $\left( b(b+1) / 2 \right)^2$ ways to choose how much of angles is cut. However, to count only valid gluings, we have to count in that in some cases the sides of angles we cut do not meet at a node of the grid, see Figure~\ref{fig:cutExB}, and that one polygon can be obtained from several rectangles, see Figure~\ref{fig:cutExC}. This yields the final formula.\end{proof}

\section{Listing the gluings}

We implemented an algorithm that lists all the gluings of at most $n$ squares for a given graph structure of a convex polyhedron. We want to point out that one gluing can admit several ways to cut itself into flat polygons. Thus it can appear in the list several times.

\begin{figure}[h] \centering
      \begin{subfigure}[b]{0.44\textwidth} \centering
            \tikz[scale=0.56]{\fill[white] (-0.4,-0.4) rectangle (4.4,4.4);
                  \foreach \i in {0,...,3} {\draw[gray,opacity=0.6] (\i,0) -- (\i,4);}
                  \foreach \i in {0,...,4} {\draw[gray,opacity=0.6] (0,\i) -- (3,\i);}
                  \draw[thick] (2,3) -- (0,0) \midnode{0} --
                        (1,1) \midnode{1} -- (3,4) \midnode{2} --
                        cycle \midnode{3};}
            \quad
            \tikz[scale=0.56]{\fill[white] (-0.4,-0.4) rectangle (4.4,4.4);
                  \foreach \i in {0,...,3} {\draw[gray,opacity=0.6] (\i,0) -- (\i,4);}
                  \foreach \i in {0,...,4} {\draw[gray,opacity=0.6] (0,\i) -- (3,\i);}
                  \draw[thick] (3,3) -- (2,4) \midnode{3} --
                        (0,1) \midnode{2} -- (1,0) \midnode{1} --
                        cycle \midnode{0};}
      \caption{} \end{subfigure} \qquad
      \begin{subfigure}[b]{0.44\textwidth} \centering
            \tikz[scale=0.56]{\fill[white] (-0.4,-0.4) rectangle (4.4,4.4);
                  \foreach \i in {0,...,3} {\draw[gray,opacity=0.6] (\i,0) -- (\i,4);}
                  \foreach \i in {0,...,4} {\draw[gray,opacity=0.6] (0,\i) -- (3,\i);}
                  \draw[dashed] (2,3)--(1,1.5) (2,2.5)--(1,1);
                  \draw[thick] (1,4) -- (1,1) \midnode{0} --
                        (2,0) \midnode{1} -- (2,3) \midnode{2} --
                        cycle \midnode{3};}
            \quad
            \tikz[scale=0.56]{\fill[white] (-0.4,-0.4) rectangle (4.4,4.4);
                  \foreach \i in {0,...,3} {\draw[gray,opacity=0.6] (\i,0) -- (\i,4);}
                  \foreach \i in {0,...,4} {\draw[gray,opacity=0.6] (0,\i) -- (3,\i);}
                  \draw[dashed] (2,4)--(1,2.5) (2,1.5)--(1,0);
                  \draw[thick] (2,4) -- (2,1) \midnode{0} --
                        (1,0) \midnode{1} -- (1,3) \midnode{2} --
                        cycle \midnode{3};}
      \caption{} \end{subfigure}
      \caption{Doubly covered parallelogram can be cut into two flat quadrilaterals in two ways, the latter consisting of its faces}
      \label{fig:twonets}
\end{figure}


\begin{example}
	A doubly covered parallelogram shown in Figure~\ref{fig:twonets} admits two ways to cut itself into two flat quadrilaterals, the latter consisting of the faces of the polyhedron.
\end{example}

\section{Algorithm to classify edge-to-edge gluings of squares}

\begin{figure}[h]
\begin{algorithmic}[1]
	\State $L(n) = \text{list of all edge-to-edge gluings of at most $n$ squares}$
	\State $M(n) = [\,]$ — the list of matrices of pairwise distances between vertices of polyhedra
	\ForAll{$\Gamma \in L(n)$}
	    \ForAll{$v_i$ — vertex of $\Gamma$ of nonzero curvature}
		\State $(M(n))_i = \text{\scshape Chen-Han} (\Gamma,v_i)$
	    \EndFor
	\EndFor
	\State Unicalize $M(n)$ up to homothety and permutation of lines, leave only corresponding elements of $L(n)$
\end{algorithmic}
	\caption{Listing and classification of edge-to-edge gluings of squares}
	\label{fig:alglistall}
\end{figure}

Due to Theorems~\ref{thm:n36}, generating $L(n)$ takes polynomial time, and $L(n)$ contains polynomial number of instances. Due to Theorem~\ref{thm:shortestSquare}, it takes $O(n^2)$ time to compute $(M(n))_i$ at $O(n^3)$ time to compute $M(n)$. Since $M(n)$ contains matrices of dimensions at most $8 \times 8$, it takes polynomial time to unicalize $M(n)$, which will result in listing all non-isomorphic edge-to-edge gluings of at most $n$ squares.

\section{Discussion}

The cornerstone of the technique we have been using is the possibility to draw a face of a polyhedra glued from squares on a planar grid. It allowed us to estimate the number of valid gluings. The same technique can seemingly be applied for the cases of regular hexagons and triangles, since these polygons also form a planar grid.

\bibliography{boris-bac}

\end{document}
