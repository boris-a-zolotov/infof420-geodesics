\documentclass[a4paper,UKenglish,cleveref]{lipics-v2021}

%%---------------------------------------------------------------------------%%
% always generating inline graphics
%\usepackage[pdftex]{graphicx}                       % must be given before tikz
%\usepackage{tikz}
%\usetikzlibrary{decorations.pathmorphing}
%\AtBeginDocument{\long\def\beginpgfgraphicnamed#1#2\endpgfgraphicnamed{#2}}
%\newlength{\unit}

% generating inline graphics unless external graphic exists
% externalising graphics: pdflatex --jobname=graphicname semi.tex
% must set correct \pgfrealjobname!!!
%\usepackage{tikz}
%\usetikzlibrary{decorations.pathmorphing}
%\pgfrealjobname{gc-tmatch}
%\newlength{\unit}

% always using external graphics, pgf/tikz not loaded
% \long\def\beginpgfgraphicnamed#1#2\endpgfgraphicnamed{\includegraphics[scale=\gscale]{#1}}
%\usepackage[pdftex]{graphicx}
% \usepackage{graphicx}

%%---------------------------------------------------------------------------%%
%% AMS-LaTeX package

\usepackage{amsmath}
\allowdisplaybreaks
% \usepackage{atmath}
% \usepackage{amssymb}

%%---------------------------------------------------------------------------%%


%%---------------------------------------------------------------------------%%
% \usepackage{amsthm,amsthm_ext}

% \usepackage[amsmath,thmmarks]{ntheorem}
% \usepackage{ntheorem_extbig}

% \usepackage[capitalise,noabbrev]{cleveref}

%%---------------------------------------------------------------------------%%
\newcommand{\Half}[1][1]{\frac{#1}{2}}
\newcommand{\tHalf}[1][1]{\tfrac{#1}{2}}

\newcommand{\R}{\mathsf}
\newcommand{\C}{\mathrm}
\newcommand{\T}{\mathfrak}
\newcommand{\W}{\mathcal}

\newcommand{\A}{\hat}

\newcommand{\idx}{\mathsf{id}}

\newcommand{\sxpx}{\mathit{sx.px}}
\newcommand{\subs}{\mathit{sub.s}}
\newcommand{\ssub}{\mathit{s.sub}}
\newcommand{\pxsx}{\mathit{px.sx}}

\newcommand{\bimati}{%
  \bigpa{\begin{smallmatrix} 1 & 0 \\ 0 & 1 \end{smallmatrix}}}

\newcommand{\bimatii}{%
  \bigpa{\begin{smallmatrix} 0 & 1 \\ 1 & 0 \end{smallmatrix}}}

\newcommand{\why}[1]{\tag*{\small (#1)}}

\newcommand{\charguard}{\textsf{\$}}
% \newcommand{\charwild}{\textsf{\upshape ?}}
\newcommand{\charwild}{\textsf ?}
%\newcommand{\charwild}{\textit{?}}

\DeclareMathSymbol{:}{\mathpunct}{operators}{"3A}

\newcommand{\pa}[1]{(#1)}
\newcommand{\bigpa}[1]{\bigl(#1\bigr)}
\newcommand{\bra}[1]{[#1]}
\newcommand{\bigbra}[1]{\bigl[#1\bigr]}
\newcommand{\brc}[1]{\{#1\}}
\newcommand{\bigbrc}[1]{\bigl\{#1\bigr\}}
\newcommand{\ang}[1]{\langle#1\rangle}
\newcommand{\bigang}[1]{\bigl\langle#1\bigr\rangle} 

\newcommand{\becomes}[1][]{\overset{\thickspace #1}{\leftarrow}}

% \newcommand{\rG}{\mathsf G}
\newcommand{\rH}{\mathsf H}
\newcommand{\rP}{\mathsf P}
% \newcommand{\cG}{\mathcal G}
% \newcommand{\cH}{\mathcal H}
% \newcommand{\cS}{\mathcal S}

% \newcommand{\A}{\hat}
% \newcommand{\T}{\mathfrak}

% \newcommand{\ssub}{\mathit{s.sub}}
\newcommand{\SSub}[1]{#1^\ssub}

% \newcommand{\charguard}{\textsf{\$}}
% \newcommand{\charwild}{\textsf{\upshape ?}}

%%---------------------------------------------------------------------------%%
%% miscellaneous packages

% \usepackage{subfig}
% \usepackage{verbatim,url}
% \usepackage{ifthen}

% \renewcommand{\topfraction}{0.9}
% \renewcommand{\textfraction}{0.1}

% \newcommand{\option}{\AtEndDocument}
% \newcommand{\option}[1]{}

%%===========================================================================%%

% \EventEditors{Katharina T. Huber and Dan Gusfield}
% \EventNoEds{2}
% \EventLongTitle{19th International Workshop on Algorithms in Bioinformatics (WABI 2019)}
% \EventShortTitle{WABI 2019}
% \EventAcronym{WABI}
% \EventYear{2019}
% \EventDate{September 8--10, 2019}
% \EventLocation{Niagara Falls, NY, USA}
% \EventLogo{}
% \SeriesVolume{143}
% \ArticleNo{16}

\ccsdesc[500]{Theory of computation~Pattern matching}
\ccsdesc[500]{Theory of computation~Algorithm design techniques}
\ccsdesc[500]{Theory of computation~Dynamic programming}
\ccsdesc[500]{Theory of computation~Divide and conquer}

\begin{document}

% \newcommand{\mytitle}{%
\title{A local edit distance oracle: Simpler, smaller, faster to construct}

\titlerunning{A local edit distance oracle}

\newcommand{\myabstract}{%
The longest common subsequence (LCS) and the edit distance (ED) problems 
are two classical string comparison problems.
Their generalisations, asking for local comparisons 
between designated subsets of substrings of the input strings,
also are of considerable theoretical and practical importance.
In particular, a local LCS (or ED) oracle is defined as a data structure
that can be precomputed from the input string pair,
and can answer queries for the LCS length (respectively, ED) 
between any designated pair of substrings from each string.
In this paper, we propose a new LCS/ED oracle 
with construction time and space $\tilde O(n^2)$ and query time $\tilde O(1)$,
thus achieving optimality within a polylogarithmic factor 
simultaneously in construction time (assuming SETH) and query time.
% This improves on the ED oracle by Charalampopoulos et al.\ \cite{Ch} in construction time and space, 
% and also simplifies their construction.
}

\author{Alexander Tiskin}{Department of Mathematics and Computer Science, 
St.~Petersburg University, 
Russia}{}{}{}

\authorrunning{A.~Tiskin}

\Copyright{Alexander Tiskin}

\keywords{longest common subsequence;
string comparison;
maximum clique;
circle graphs}

\maketitle

\begin{abstract}
\myabstract
\end{abstract}

% \renewcommand{\mythanks}{%
% Research supported by DIMAP, University of Warwick.}

%%===========================================================================%%

%%---------------------------------------------------------------------------%%
\section{Introduction}

The longest common subsequence (LCS) and the edit distance (ED) problems 
are two classical string comparison problems, see e.g.\ \cite{xxx}.
Given a pair of strings $a$, $b$, the \emph{LCS problem}
asks for the length of the longest string that occurs 
as a (not necessarily contiguous) subsequence in both $a$, $b$.
For simplicity, we let $a$, $b$ be of equal length $n$.
The \emph{ED problem} asks for the minimum number of edits that are required to transform $a$ into $b$,
where an edit is defined either as a character insertion or deletion 
(this version of the problem is called 
\emph{indel distance}, \emph{LCS distance}, or \emph{Feldman--Katok distance}),
or can additionally include character substitution
(\emph{indelsub distance}, or \emph{Levenshtein distance}).

These problems' generalisations, asking for local comparisons 
between designated subsets of substrings of the input strings,
also are of considerable theoretical and practical importance.
For instance, approximate pattern matching \cite{xxx}
compares a fixed pattern string to the substrings of the text string,
and can therefore be termed \emph{semi-local comparison}.
One of the main algorithmic techniques in computational molecular biology, 
the Smith--Waterman algorithm \cite{xxx},
compares substrings in both input strings, and returns the most similar pair of substrings from each string
(or, more generally, for each pair of the input strings' prefixes, returns their most similar suffixes),
and therefore represents \emph{local comparison}.
A more general, and more difficult, form of local comparison 
asks for an (implicit) comparison between all pairs of substrings from each string,
not necessarily the most similar such pairs either locally or globally.
In particular, a \emph{local LCS} (or \emph{ED}) \emph{oracle} is defined as a data structure
that can be precomputed from the input string pair,
and can answer queries for the LCS length (respectively, ED) 
between any designated pair of substrings from each string.
A \emph{semi-local LCS} (or \emph{ED}) \emph{oracle} is similar,
but restricted to queries between a whole input string and a designated substring of the other input string,
and/or queries between a designated prefix of one input string and a designated suffix of the other.

\begin{table}\centering
\begin{tabular}{|l|l|l|l|}\hline
%
Reference & constr. & space & query \\ \hline
%
classical DP ($n$ times) & $O(n^3)$ & $O(n^2)$ & \\
%
Schmidt \cite{Schmidt:98}  & $\tilde O(n^2)$ & $O(n^2)$ & \\
%
Alves et al.\ \cite{Alves+:08}, Tiskin \cite{Tiskin:06_CSR} & $O(n^2)$ & $O(n^2)$ & \\ \hline
%
Alves et al.\ \cite{Alves+:08}, Tiskin \cite{Tiskin:06_CSR} & $O(n^2)$ & $O(n)$ & $O(l)$ \\
%
Tiskin \cite{Tiskin:06_CSR} & $O(n^2)$ & $\tilde O(n)$ & $\tilde O(1)$ \\ \hline
%
\end{tabular}
%
\caption{\label{t-slcs}Semi-local LCS/ED: algorithms (with $O(n^2)$ explicit output) and oracles}
%
\end{table}

\begin{table}\centering
\begin{tabular}{|l|l|l|l|}\hline
%
Reference & constr. & space & query \\ \hline
%
classical DP ($n^2$ times) & $O(n^4)$ & $O(n^4)$ & \\ \hline
%
Schmidt \cite{Schmidt:98} ($n$ times), Tiskin \cite{Tiskin:06_CSR} ($n$ times) & 
$\tilde O(n^3)$ & $\tilde O(n^3)$ & $\tilde O(1)$ \\
%
Tiskin \cite{Tiskin:08}, Sakai \cite{Sakai:19}, Charalampopoulos et al.\ \cite{Charalampopoulos+:20} & 
$\tilde O(n^2)$ & $\tilde O(n^2)$ & $O(l)$ \\
%
Sakai \cite{Sakai:22} & $\tilde O(n^2)$ & $\tilde O(n^2)$ & $O(l^{1/2})$ \\
%
Charalampopoulos et al.\ \cite{Charalampopoulos+:21} & 
$O(n^{2+\epsilon})$ & $O(n^{2+\epsilon})$ & $\tilde O(1)$ \\
%
this work & $\tilde O(n^2)$ & $\tilde O(n^2)$ & $\tilde O(1)$ \\ \hline
%
\end{tabular}
%
\caption{\label{t-llcs}Local LCS/ED: algorithms (with $O(n^4)$ explicit output) and oracles}
%
\end{table}

The classical dynamic programming (DP) algorithm for the (global) LCS/ED problem
was rediscovered independently many times; see \cite{jeffe} for a timeline and references.
The algorithm in fact solves the LCS/ED problem between every pair of prefixes of $a$, $b$.
Therefore, one can solve the semi-local (respectively, local) LCS/ED problem naively,
by applying the classical DP algorithm repeatedly $O(n)$ (respectively, $O(n^2)$) times
to the original $a$ against all suffixes of $b$ 
(respectively, to all pairs of suffixes of $a$, $b$).

Schmidt \cite{Schmidt:98} made the first important insights,
leading to an improvement on the naive approach.
In particular, she gave an oracle for solving the LCS/ED problem 
between all prefixes of string $a$ and all substrings of string $a$,
with construction time $\tilde O(n^2)$ and query time $\tilde O(1)$.
This oracle also serves as an algorithm for the semi-local LCS/ED problem running in time $\tilde O(n^2)$.
One can obtain a local LCS/ED oracle by constructing the oracle of \cite{Schmidt:98} 
repeatedly $O(n)$ times for all suffixes of $a$ against the original $b$.
Alves et al.\ \cite{Alves+:08} gave an algorithm for the semi-local LCS/ED problem running in time $O(n^2)$,
and a semi-local LCS/ED oracle with construction time $O(n^2)$ and query time $O(l)$,
where $l$ is the length of the query substring (for string-substring LCS/ED).
This was improved by a polylogarithmic factor, and also extended to prefix-suffix LCS/ED
(where $l$ is the sum of the lengths of the query prefix and query suffix)
by the current author \cite{Tiskin:06_CSR}.
Further, the current author \cite{Tiskin:06_CSR,Tiskin:08} exposed a deep connection 
between the semi-local LCS/ED problem and an algebraic structure known as the \emph{Hecke monoid},
and gave a semi-local LCS/ED oracle
with construction time $\tilde O(n^2)$ and query time $\tilde O(1)$.
Similarly to the oracle of \cite{Schmidt:98}, the oracle of \cite{Tiskin:06_CSR,Tiskin:08}
can be used to obtain a local LCS/ED oracle, by constructing it
repeatedly $O(n)$ times for all suffixes of $a$ against the original $b$.

Another improvement to local LCS/ED oracles comes by constructing semi-local oracles
for a hierarchical partitioning of string $a$ against the original string $b$.
This technique was introduced and applied in different forms 
by the current author \cite{Tiskin:08}, Sakai \cite{Sakai:19}, 
Charalampopoulos et al.\ \cite{Charalampopoulos+:20}.
The resulting local LCS/ED oracle
has construction time $\tilde O(n^2)$ and query time $O(l)$,
where $l$ is the sum of the lengths of query substring pair.
More recently, substantial breakthroughs have been made by Sakai \cite{Sakai:22},
who gave a local LCS/ED oracle
with construction time $\tilde O(n^2)$ and query time $O(l^{1/2})$,
and by Charalampopoulos et al.\ \cite{Charalampopoulos+:20},
who gave a local LCS/ED oracle
with construction time $O(n^{2+\epsilon})$ and query time $O(1)$.
\cref{t-slcs,t-llcs} list the aforementioned results
(except for the polylogarithmic speedups, which we consider to be outside the scope for this paper).

Besides the generic algorithms and oracles for the semi-local and local LCS/ED problems,
a number of solutions for restricted versions of these problems have been proposed.
In particular, this author \cite{Tiskin:08_MCS} and Krusche and Tiskin \cite{Krusche_Tiskin:10}
gave algorithms for the window-substring and window-window LCS problems, running in time $O(n^2)$;
here, the term \emph{window} refers to any substring of $a$ of some fixed length.
Sakain {Sakai:11} gave an algorithm for the general fragment-substring problem, running in time $O(n^2)$;
here, the term \emph{window} refers to any substring of $a$ 
from some fixed set of substrings of cardinality $O(n)$.
Barsky et al.\ \cite{Barsky+:08} gave a parameterised algorithm for obtaining
all maximal substring pairs within edit distence $\kappa$, running in time $O(n^2 \kappa^3)$.

The currently best known local LCS oracle, in terms of $\tilde O$ asymptotics, 
is the one by Charalampopoulos et al.\ \cite{Charalampopoulos+:21}.
The description of this oracle is quite intricate,
using a weighted Voronoi partitioning of the LCS grid, with complex bookkeeping and parameter tuning.
Underneath this LCS/ED-specific layer, this oracle relies on a layer of data structures and subroutines 
for the multiple-source shortest paths (MSSP) problem in a planar graph, developed by Klein \cite{Klein}.

In this paper, we propose a new LCS/ED oracle 
with construction time and space $\tilde O(n^2)$ and query time $\tilde O(1)$.
We replace the lower layer of generic planar MSSP
by a simpler, more specific framework of data structures and subroutines
for the semi-local LCS problem developed by Tiskin \cite{Tiskin}.
Apart from providing a significant simplification by itself,
the use of this framework also simplifies and streamlines the upper layer of weighted Voronoi diagrams.
These modifications also result in an improvement on the ED oracle by Charalampopoulos et al.\ \cite{Ch} 
to construction time and space $\tilde O(n^2)$, while keeping query time $\tilde O(1)$.
Considering the LCS lower bound of \cite{xxx},
our oracle is optimal to within polylogarithmic factors, assuming the Strong Exponential Time Hypothesis.

%%---------------------------------------------------------------------------%%
\section{The framework}
\label{s-framework}

In this section, we introduce briefly the framework developed by the author 
in \cite{Tiskin:08_MCS,Tiskin:08_JDA,Tiskin:09_JMS,Tiskin:15_Algorithmica}.
Definitions and results of this section have either appeared in these publications,
or are their straightforward generalisations.

%%---------------------------------------------------------------------------%%
\subsection{Preliminaries}

For matrix and vector indices, we will use either integers, or half-integers%
\footnote{The intuition behind using both integers and half-integers
is that when comparing a pair of strings, 
we are dealing with a planar grid-like graph;
our techniques also require dealing with its dual graph.
In this setting, it is natural to use integers for indexing the nodes of the primal grid,
and half-integers for indexing the nodes of the dual grid (i.e.\ the faces of the primal grid).}
(sometimes called odd half-integers):
%
\begin{gather*}
%
\brc{\ldots, -2, -1, 0, 1, 2, \ldots}\qquad
\bigbrc{\ldots, -\tHalf[5], -\tHalf[3], -\tHalf[1], 
 \tHalf[1], \tHalf[3], \tHalf[5], \ldots}
%
\end{gather*}
%
For ease of reading, half-integer variables will be indicated by hats
(e.g.\ $\A\imath$, $\A\jmath$).
Ordinary variable names 
(e.g.\ $i$, $j$, with possible subscripts or superscripts), 
will normally denote integer variables,
but can sometimes denote a variable 
that may be either integer, or half-integer.

It will be convenient to denote
%
\begin{gather*}
%
i^- = i-\tHalf \qquad i^+ = i+\tHalf
%
\end{gather*}
%
for any integer or half-integer $i$.
The set of all half-integers can now be written as
%
\begin{gather*}
%
\bigbrc{\ldots, (-3)^+, (-2)^+, (-1)^+, 0^+, 1^+, 2^+, \ldots}
%
\end{gather*}

\index{interval notation}%
We denote integer and half-integer \emph{intervals} by
%
\begin{gather*}
%
\bra{i:j} = \brc{i, i+1, \ldots, j-1, j} \qquad
\ang{i:j} = \bigbrc{i^+, i+\tHalf[3], \ldots, j-\tHalf[3], j^-}
%
\end{gather*}
%
In both cases, the interval is defined by a pair of integer \emph{endpoints}.
We will use round parentheses $(i:j)$ to indicate that in the given context,
either square $\bra{i:j}$ or angle $\ang{i:j}$ brackets may be used.

For finite intervals $\bra{i:j}$ and $\ang{i:j}$,
we call the difference $j-i$ interval \emph{length}.
Note that for a given length $n$, 
an integer interval consists of $n+1$ elements,
and a half-integer interval of $n$ elements.

\index{$\pa{i_0:i_1 ; j_0:j_1}$: interval Cartesian product}%
We use the semicolon notation for the Cartesian product 
of two integer or two half-integer intervals:
%
\begin{gather*}
%
\bra{i:j ; k:l} = \bra{i:j} \times \bra{k:l}\qquad
\ang{i:j ; k:l} = \ang{i:j} \times \ang{k:l}
%
\end{gather*}
%
We identify a singleton interval with its only element, 
extending this notation to Cartesian products.
Thus, we have, for example
%
% We will denote a singleton Cartesian product (i.e.\ an ordered pair) 
% by $\bra{i,j} = \bra{i ; j}$, $\ang{i,j} = \ang{i ; j}$.
%
\begin{gather*}
%
\bra{i;j} = \bra{i:i ; j:j}\qquad
\ang{\A\imath ; j:k} = \ang{\A\imath^- : \A\imath^+ ; j:k}
%
\end{gather*}

We indicate the index range of a matrix by juxtaposition, 
e.g.\ matrix $A\pa{i:i' \mid j:j'}$ over an integer or half-integer index range.
The same notation will be used for selecting subvectors and submatrices:
for example, given matrix $A\pa{0:n \mid 0:n}$, 
we denote by $A\pa{i:i' \mid j:j'}$
the submatrix defined by the given sub-intervals.
When any of the indices $i$, $i'$, $j$, $j'$ coincide with the range boundary,
they can be omitted,
e.g.\ $A\pa{i: \mid :j'} = A\pa{i:n \mid 0:j'}$,
and $A\pa{: \mid j:j'} = A\pa{0:n \mid j:j'}$.
In particular, we write $A\pa{i,j}$ for a single matrix element,
$A\pa{i \mid :}$ for a row, and $A\pa{: \mid j}$ for a column.
We will denote by $\sum A$ the sum of all elements in matrix $A$.

By default, vectors and matrices will be indexed by integers based at $0$, 
or by half-integers based at $0^+ = \tHalf$.
Where necessary, all our definitions and statements can easily be generalised
to indexing over arbitrary integer or half-integer intervals.

Given a string, we distinguish between its contiguous \emph{substrings},
and not necessarily contiguous \emph{subsequences}.
Special cases of a substring 
are \emph{a prefix} and \emph{a suffix} of a string.
Given string $a$ and a set $I$ of indices,
we denote by $a|_I$ a subsequence formed by the characters at the specified indices.

Unless indicated otherwise, an algorithm's input is
a string $a$ of length $m$, and a string $b$ of length $n$;
we assume $m \leq n$.
Normally, we will say that two characters $\alpha$, $\beta$ \emph{match}, 
if $\alpha=\beta$, and \emph{mismatch} otherwise.
In addition to this, we introduce the \emph{wildcard character} `\charwild',
which matches itself and all other characters.

\begin{definition}
\label{def-dsum}
%
Let $D\ang{0:n_1 \mid 0:n_2}$ be a matrix.
Its \emph{dominance-sum matrix} (also called \emph{distribution matrix})
$D^\Sigma\bra{0:n_1 \mid 0:n_2}$ is defined by
%
\begin{gather*}
%
\textstyle
D^\Sigma\bra{i,j} = \sum D\ang{i: \mid :j}
%
\end{gather*}
%
for all $i \in \bra{0:n_1}$, $j \in \bra{0:n_2}$.
%
\end{definition}

\begin{definition}
\label{def-xdiff}
%
Let $A\bra{0:n_1 \mid 0:n_2}$ be a matrix.
Its \emph{cross-difference matrix} $A^\square\ang{0:n_1 \mid 0:n_2}$
(also called \emph{density matrix}) is defined by
%
\begin{gather*}
%
A^\square\ang{\A\imath,\A\jmath} =
A\bra{\A\imath^+,\A\jmath^-} - A\bra{\A\imath^-,\A\jmath^-} -
A\bra{\A\imath^+,\A\jmath^+} + A\bra{\A\imath^-,\A\jmath^+}
%
\end{gather*}
%
for all $\A\imath \in \ang{0:n_1}$, $\A\jmath \in \ang{0:n_2}$.
%
\end{definition}

\begin{definition}
\label{def-monge}
%
\index{matrix!Monge}%
Matrix $A$ is a \emph{Monge matrix},
if its cross-difference matrix $A^\square$ is nonnegative.
Matrix $A$ is an \emph{anti-Monge matrix}, if its negative $-A$ is Monge.
%
\end{definition}

\begin{definition}
\label{def-permutation}
%
A \emph{permutation matrix}
is a zero-one matrix containing exactly one nonzero 
in every row and every column.
%
\end{definition}

\begin{definition}
\label{def-umonge}
%
Matrix $A$ is called \emph{unit-Monge},
if its cross-difference matrix $A^\square$ is a permutation matrix.
Matrix $A$ is called \emph{unit-anti-Monge}, 
if its negative $-A$ is unit-Monge.
%
\end{definition}

\begin{definition}
\label{def-simple}
%
Matrix $A$ is called \emph{simple},
if its leftmost column (the one with the lowest index) 
and bottom row (the one with the highest index) are identically zero.
%
\end{definition}

\begin{theorem}
\label{th-umonge-matrix-query}
%
Given a permutation matrix $P$ with $n$ nonzeros, 
there exists a data structure that
%
\begin{itemize}
%
\item has size $O\bigpa{n \log n}$;
\item can be built in time $O\bigpa{n \log n}$;
\index{matrix!implicit!element query}%
\item supports queries for any individual element 
of the simple (sub)unit-Monge matrix $P^\Sigma$ in time $O\bigpa{\log^2 n}$;
%
\end{itemize}
%
\end{theorem}

\begin{theorem}
\label{th-umonge-matrix-query-inc}
%
Given a permutation matrix $P$ and the value $P^\Sigma[i;j]$,
the values $P^\Sigma[i \pm 1;j]$, $P^\Sigma[i;j \pm 1]$,
where they exist, can be queried in time $O(1)$.
%
\end{theorem}

%%---------------------------------------------------------------------------%%
\subsection{Sticky matrix multiplication}

In our previous work, we established the connection between string comparison
and implicit distance multiplication of unit-Monge matrices,
expressed as \emph{sticky multiplication} $P \boxdot Q = R$ 
on the corresponding permutation matrices.
%
Subquadratic distance multiplication algorithms 
for implicit simple unit-Monge matrices 
were given in \cite{Tiskin:06_CSR,Tiskin:08_MCS}, 
and culminated with the following result in \cite{Tiskin:15_Algorithmica}.
%
\begin{theorem}
\label{th-pmat-smult}
%
Let $P$, $Q$, $R$ be permutation matrices with at most $n$ nonzeros each.
Let $P \boxdot Q = R$.
Given the nonzeros of $P$, $Q$, the nonzeros of $R$ can be computed in time $O(n \log n)$.
%
\end{theorem}


%%---------------------------------------------------------------------------%%
\subsection{LCS and semi-local LCS}

Unless indicated otherwise, a string comparison problem will take as input 
strings $a\ang{0:m}$ and a string $b\ang{0:n}$ (indexed by half-integers).
%
\begin{definition}
\label{def-lcs}
%
Let $a$, $b$ be strings.
The \emph{longest common subsequence (LCS) score} $\mathsf{score}_{LCS}(a,b)$
is the length of the longest string that is a subsequence of both $a$ and $b$.
Given strings $a$, $b$, the \emph{LCS problem} asks for the LCS score $\mathsf{score}_{LCS}(a,b)$.
%
\end{definition}
%
The classical dynamic programming algorithm for the LCS problem 
\cite{Needleman_Wunsch:70,Wagner_Fischer:74} runs in time $O(mn)$.
The best known algorithms speed it up by a (model-dependent) polylogarithmic factor
% \cite{Masek_Paterson:80,Crochemore+:03_SIAM,Wu+:96,Bille_Farach:08}.
\cite{Masek_Paterson:80,Crochemore+:03_SIAM,Bille_Farach:08}.
Similar speedups are possible for the algorithms presented in this paper;
however, we will consider them outside the paper's scope, 
and will not discuss them any further.

Although global comparison (full string against full string)
and fully-local comparison (all substrings against all substrings)
are the two most common approaches to comparing strings,
another important type of string comparison lies ``in between''.
%
\begin{definition}
\label{def-slcs}
%
Given strings $a$, $b$, the \emph{semi-local LCS problem}
asks for the LCS scores as follows:
%
\begin{itemize}
%
\item the whole $a$ against every substring of $b$
(\emph{string-substring LCS});
%
\item every prefix of $a$ against every suffix of $b$
(\emph{prefix-suffix LCS});
%
\item every suffix of $a$ against every prefix of $b$
(\emph{suffix-prefix LCS});
%
\item every substring of $a$ against the whole $b$
(\emph{substring-string LCS}).
%
\end{itemize}
%
\end{definition}
%
Hereafter, we allow the output of the semi-local LCS problem and its variants
to be represented implicitly by an oracle 
that allows efficient querying of a score for any given substring pair as specified.

Note that \cref{def-slcs} is symmetric with respect to exchanging the two strings.
In many ways, our approach consists in exploiting to the full 
this and other symmetries of the LCS problem,
such as the symmetry between the left and the right within each of the strings.

Some alternative terms for semi-local comparison,
used especially in biological texts, 
are ``end-free alignment''
\cite{Jackson_Aluru:06}, \cite[Subsection 11.6.4]{Gusfield:97}
or ``semi-global alignment'' 
\cite{Jackson_Aluru:06}, \cite[Problem 6.24]{Jones_Pevzner:04}.
% \cite[Section 8.4]{Gogol+:10}.
The string-substring (and its symmetric substring-string) component
of semi-local string comparison is also called ``fitting alignment'' 
\cite[Problem 6.23]{Jones_Pevzner:04}.
String-substring LCS is an important problem in its own right,
closely related to approximate pattern matching,
where a short fixed pattern string 
is compared to various substrings of a long text string.

% \subsection{LCS grid.}
The LCS problem can be represented by a graph as follows.
%
\begin{definition}
\label{def-grid}
%
\index{grid!triangulated}%
A \emph{triangulated grid} $G\bra{0:m;0:n}$ is a directed acyclic graph, 
defined on the set of nodes $\bra{0:m;0:n}$.
For all $i \in \bra{0:m}$, $\A\imath \in \ang{0:m}$,
$j \in \bra{0:n}$, $\A\jmath \in \ang{0:n}$,
the triangulated grid contains:
%
\begin{itemize}
%
\item the horizontal edge $\bra{i;\A\jmath^-} \to \bra{i;\A\jmath^+}$;
%
\item the vertical edge $\bra{\A\imath^-;j} \to \bra{\A\imath^+;j}$;
%
\item the diagonal edge $\bra{\A\imath^-;\A\jmath^-} \to \bra{\A\imath^+;\A\jmath^+}$.
%
\end{itemize}
%
\end{definition}
%
We can also view such a triangulated grid as an $m \times n$ 
grid of \emph{cells}, each induced by four neigbouring nodes.
A cell contains a pair of vertical edges, a pair of horizontal edges, and a diagonal edge.

Triangulated grids representing string comparison will have weights on the edges.
These weights will be called \emph{scores}.
The score of a path is defined as the sum of the scores along its edges.
%
\begin{definition}
\label{def-grid-lcs}
%
\index{grid!LCS}%
\index{$\R G_{a,b}$: LCS grid}%
The \emph{LCS grid} for strings $a$, $b$
is a weighted triangulated grid $\R G_{a,b}\bra{0:m;0:n}$, defined as follows.
A cell induced by nodes $\bra{\A\imath^\pm;\A\jmath^\pm}$, 
$\A\imath\in\ang{0:m}$, $\A\jmath\in\ang{0:n}$, is called
%
\begin{itemize}
%
\item a \emph{match cell}, if $a\ang{\A\imath}$ matches $b\ang{\A\jmath}$;
\item a \emph{mismatch cell}, otherwise
%
\end{itemize}
%
An edge is called
%
\begin{itemize}
%
\item a \emph{match edge}, if it is a diagonal edge in a match cell;
\item a \emph{mismatch edge}, if it is a diagonal edge in a mismatch cell;
\item a \emph{gap edge}, if it is a horizontal or vertical edge.
%
\end{itemize}
%
Match edges have score $1$; mismatch and gap edges have score $0$.
Mismatch edges do not affect maximum path scores between given endpoints, 
and can therefore be ignored.
%
\end{definition}

Let $b^{\mathit{pad}}\ang{-m:m+n} = \charwild^m \underline{b} \charwild^m$.
%
\begin{definition}
\label{def-lcsmat}
%
The \emph{semi-local LCS matrix} is defined as 
$\rH_{a,b}[i,j] = \mathsf{score}_{LCS}(a, b^{\mathit{pad}}\ang{i,j})$,
where $i \in \bra{-m:n}$, $j \in \bra{0:m+n}$.
The \emph{string-substring LCS matrix} is defined as 
$\SSub\rH_{a,b}[i,j] = \rH_{a,b}[i,j] = \mathsf{score}_{LCS}(a, b\ang{i:j})$, 
where $i,j \in \bra{0:n}$.
%
\end{definition}
%
Matrix $\rH_{a,b}$ provides (subject to some easy linear transformations)
the solution to all four components of the semi-local LCS problem;
its submatrix $\SSub\rH_{a,b}$ singles out the string-substring LCS component.
In \cite{Tiskin:08_MCS}, we show that matrix $\rH_{a,b}$ is unit-anti-Monge,
% By \cref{th-monge-decomp}, it can be represented implicitly in compact form 
so it can be represented implicitly in compact form 
by a permutation matrix $\rP_{a,b}$, that we call here \emph{semi-local LCS kernel},
from which every element of $\rH_{a,b}$ (and therefore also of its submatrix $\SSub\rH_{a,b}$) 
can be queried efficiently.
The elements $\rP_{a,b}\ang{\A\imath,\A\jmath}$ are indexed 
by $\A\imath \in \ang{-m:n}$, $\A\jmath \in \ang{0:m+n}$.
In \cite{Tiskin:08_MCS}, we also give an algorithm 
computing this implicit representation in time $O(mn)$.

We will abbreviate ``semi-local LCS kernel'' to just \emph{LCS kernel}, 
where that is justified by the context.
% The key idea of our approach is to regard \cref{th-lcsmat-decomp}
The key idea of our approach is to use LCS kernels
as implicit solutions to the semi-local LCS problem.
% as defining an implicit solution to the semi-local LCS problem.

%\begin{figure}%[tbp]
%\centering
%
%\includegraphics{f-grid-final.pdf}
%
%\caption{\label{f-slcs-braid}%
%LCS grid $\R G_{a,b}$ and semi-local LCS kernel $\R P_{a,b}$ as a sticky braid}
%\end{figure}

\textcolor{magenta}{Picture {\tt f-grid-final}}

\begin{example}
%
\cref{f-slcs-braid} shows the LCS kernel $\R P_{a,b}$ for strings 
$a = \textsf{``BAABCBCA''}$, $b = \textsf{``BAABCABCABACA''}$,
in terms of a \emph{sticky braid} embedded in the LCS grid $\R G_{a,b}$.
The edges on the top-left boundary of the grid are indexed by $\A\imath \in \ang{-8:13}$
from bottom-left to top-left to top-right, 
and the edges on the bottom-right boundary of the grid by $\A\jmath \in \ang{0:8+13}$
from bottom-left to bottom-right to top-right.
The nonzeros of the LCS kernel $\R P_{a,b}$ correspond 
to those pairs of indices $\A\imath$, $\A\jmath$
that are connected by a strand in the sticky braid.
%
\end{example}
%
\begin{definition}
%
\index{problem!oracle}%
Let $X$ be an algorithmic problem 
asking for comparison scores between certain substrings of input string(s).
An \emph{$X$-oracle} with query time $T$ is a data structure obtained from the input strings,
that supports the queries specified by $X$ on substrings of length at most $s$ in time $T(s)$.
%
\end{definition}
%
\begin{theorem}
\label{th-slcs-oracle}
%
Given strings $a$, $b$, there exists a semi-local LCS oracle with 
%
\begin{itemize}
%
\item size $O(m+n)$ and query time $O(s)$;
\item size $O\bigpa{(m+n)\log(m+n)}$ and query time $O\bigpa{(\log s)^2}$;
%
\end{itemize}
%
where $s$ is the length of the shorter string or substring in the query.
%
\end{theorem}

In fact, the oracle of \cref{th-slcs-oracle} is provided by the LCS kernel $\rP_{a,b}$,
enhanced with the data structure of \cref{th-umonge-matrix-query}.

\begin{theorem}
\label{th-}
%
Let $a = a'a''$, where $a'$, $a''$ are of length $m'$, $m''$ respectively.
Let $m^* = \min(m',m'',n)$.
Given LCS kernels $\rP_{a',b}$, $\rP_{a'',b}$, the LCS kernel $\rP_{a,b}$
can be obtained in time $O\bigpa{n \log m^*}$.
%
\end{theorem}

The idea of the proof of \cref{th-}, presented in our previous work, is as follows:
if $m^* = n$, then the problem is solved by a single call 
to the algorithm of \cref{th-pmat-smult}.
Otherwise, let without loss of generality $m^* = m'$.
The problem is then solved by partitioning the LCS grid $\R G_{a',b}$
horizontally into $n/m'$ near-square blocks, and calling 
the algorithm of \cref{th-pmat-smult} once for each block.

\begin{theorem}
\label{th-slcs-mbs}
%
Given strings $a$, $b$, and assuming $m \leq n$, the LCS kernel $\rP_{a,b}$
can be obtained in time $O\bigpa{\frac{mn (\log\log n)^2}{(\log n)^2} + n}$.
%
\end{theorem}

%%---------------------------------------------------------------------------%%
\subsection{Alignment and ED}
\label{ss-align}

Blow-up

%%=-=-=-=-=-=-=-=-=-=-=-=-=-=-=-=-=-=-=-=-=-=-=-=-=-=-=-=-=-=-=-=-=-=-=-=-=-=%%
\section{The local LCS/ED oracle}
\label{s-llcs}

%%---------------------------------------------------------------------------%%
\subsection{Auxiliary structures}

Forward braid: information on highest-scoring paths originating at base level

A strand: unit obstruction to LCS

Main property: a highest-scoring path may not cross a strand twice

In other words, only ``necessary crossings'' between a path and a strand

A necessary, but not a sufficient condition.
Instead, a certain type of ``convexity'': 
a path is between leftmost and rightmost highest-scoring paths, if it never crosses a strand twice.

Double-crossing type: left-right-left or right-left-right

(dominance counting oracle, crossing oracle: 2D range tree or Patrascu)

(entanglement = double-crossing)

\emph{entanglement counting oracle}.
Set of $N$ index triples $(i, j, k)$.
Answers queries: given an arbitrary index triple, 
count how many triples in set are entangled with it (LRL, RLR).
3D range tree (or Patrascu-like).
Size $O(n \log n)$, query $O((\log n)^3)$

Points on horizontal edges.

Reduced embedded braid.
Point triple double-crossed crossed by a strand, 

\emph{strand entanglement oracle}.
Reduced embedded braid.
Answers queries: given an arbitrary point triple $(0,r), (k,s), (l,t)$,
decide whether it is double-crossed by any strand.

Partition grid hierarchically into canonical strips.
For every canonical strip (u,v), build entanglement counting oracle for levels (0, u, v).

Query: decompose strip $(k,l)$ into canonical strips.
Trace left and right boundaries of entanglement-free region at decomposition levels.
Each level: binary search with the entanglement counting oracle.
Since braid is reduced, direction for binary search is always well-defined.

Decide at level $k$: point $(k,s)$ left, within, or right of the boundaries

%%---------------------------------------------------------------------------%%
\subsection{Structure of the local LCS/ED oracle}

Oracle is built recursively, based on a hierarchical partitioning of string $a$.
At the top level, a structure to answer queries where the substring of $a$ crosses its midpoint,
beginning in the lower half and ending in the upper half of $a$.
The construction is then repeated recursively for each of the two halves of $a$.

We now describe the top level of the oracle's hierarchical construction.
Let us fix initially $l$ as the base level.
We will call $n/2$ site level, and all its nodes will be called sites.
Consider a forward braid for the subgrid below $l$.
Further consider a subgrid below $n/2$.
This subgrid is partitioned into intercept regions.
Simplified analogue of weighted Voronoi regions of Charalampopoulos et al 
(the concept, in its turn, inspired by graph oracles).

Call a node simply reachable, if it is reachable from $s$ 
without crossing from left to right any strand of the forward braid.
Intercept region for a given site $s$: consists of all grid nodes below site level
\begin{itemize}
\item simply reachable from $s$ 
\item not simply reachable from any site to the left of $s$
\end{itemize}

Partitioning achieved by terminating every strand at its first horizontal crossing.
Call this crossing the region's intercept corner.
Intercept region's left boundary formed by a full strand from site level to intercept,
where it is terminated.
Intercept region's right boundary formed by a sequence of strand pieces,
where each piece begins by a strand crossing the previous piece from right to left,
terminating the corresponding strand,
and ends by the piece being terminated by the next piece in the sequence.
The final piece is terminated at the region's intercept.

The local oracle consists of 
\begin{itemize}
\item the forward braid for the lower subgrid 
(note that this is different from the forward braids defining the intercept regions)
\item the interception oracle built on this braid;
\item for every base level, the sequence of intercepts (indexed by sites)
\end{itemize}

Size $\tilde O(n^2)$. Recursively, size $\tilde O(n^2)$.

%%=-=-=-=-=-=-=-=-=-=-=-=-=-=-=-=-=-=-=-=-=-=-=-=-=-=-=-=-=-=-=-=-=-=-=-=-=-=%%
\section{Query algorithm}
\label{s-query}




%%=-=-=-=-=-=-=-=-=-=-=-=-=-=-=-=-=-=-=-=-=-=-=-=-=-=-=-=-=-=-=-=-=-=-=-=-=-=%%
\section{Construction algorithm}
\label{s-construction}

%%=-=-=-=-=-=-=-=-=-=-=-=-=-=-=-=-=-=-=-=-=-=-=-=-=-=-=-=-=-=-=-=-=-=-=-=-=-=%%
\section{Conclusion and open problems}
\label{s-conclusion}

%%===========================================================================%%

% \bibliographystyle{plainurl}
% \bibliography{research}

\textcolor{magenta}{Bibliography!}

% \appendix
% \clearpage
% \ifthenelse{\equal{\option{x}}{}}{}{\section{Proofs and examples}}

%===========================================================================%%
\end{document}
