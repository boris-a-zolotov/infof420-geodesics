\begin{figure}[h] \centering
\begin{subfigure}[t]{3.6cm} \centering
\tikz[scale=0.45]{
	\foreach \i in {-1,...,6} {
		\draw[gray] (-1,\i) -- (6,\i) (\i,-1) -- (\i,6);
	}
	\draw[very thick,red] (4,5) -- (5,2);
	\draw[thick] (5,2) -- ++(-2,-2) -- ++(-3,0) --
		++(0,2) -- ++(1,3) -- ++(3,0);
} \caption{} \label{fig:emA} \end{subfigure} \hspace{0.6cm}
\begin{subfigure}[t]{3.6cm} \centering
\tikz[scale=0.45]{
	\foreach \i in {-1,...,6} {
		\draw[gray] (-1,\i) -- (6,\i) (\i,-1) -- (\i,6);
	}
	\draw[very thick,red] (0,4) -- (3,5);
	\draw[thick] (3,5) -- ++(2,-2) -- ++(0,-3) --
		++(-3,0) -- ++(-2,2) -- ++(0,2);
} \caption{} \label{fig:emB} \end{subfigure} \hspace{0.8cm}
\begin{subfigure}[t]{3.2cm} \centering
\tikz[scale=0.45]{
	\foreach \i in {0,...,6} {
		\draw[gray] (0,\i) -- (6,\i) (\i,0) -- (\i,6);
	}
	\draw[thick] (1,0)--(2,3);
	\draw[thick,red,dashed] (4,6)--(2,3)--(5,1);
} \caption{} \label{fig:emC} \end{subfigure}

\caption{(a), (b) Highlighted edge has the same lengths of projections on
	the drawings of two faces. (c) Two ways to place an edge with given
	projections that preserve convexity of the face.}
\end{figure}
