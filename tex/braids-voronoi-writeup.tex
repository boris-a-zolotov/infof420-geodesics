\documentclass[a4paper,11pt]{article}
\usepackage[margin=1.1in]{geometry}
\usepackage{f420}
\usepackage{modules/braid}

\title{ }
\author{ }
\date{\today}

\begin{document} \maketitle

\section{Definitions}

\begin{definition}
	\emph{A Voronoi diagram} is a subdivision of a \(n \times m\) table equipped with a greedy braid into regions such that
   \begin{enumerate}
	\item Left boundary of each region is composed of a single braid strand and possibly a piece of the boundary;
	\item Right boundary of each region is composed of several pieces \(e_1, e_2, \ldots, e_k\) of braid strands, such that \begin{enumerate}
	   \item the pieces \(e_1, e_2, \ldots, e_k\) cross no strand from left to right,
	   \item the strand containing \(e_i\) crosses the strand containing \(e_{i+1}\) from left to right at the point where \(e_i\) meets \(e_{i+1}\), \(i = 1,\ldots, k-1\).
	\end{enumerate}
   \end{enumerate}
\end{definition}

\begin{lemma}
	The following are equivalent:
   \begin{enumerate}
	\item Point \(p\) belongs to the Voronoi cell \(f_i\) of a Voronoi site \(s_i\);
	\item Site \(s_i\) is the leftmost site such that there is a path from \(s_i\) to \(p\) that crosses no braid strand from left to right.
   \end{enumerate}
\end{lemma}

\section{Notation}

\newcommand{\VD}{\ensuremath{\text{\sffamily VD}}\xspace}
\newcommand{\braid}{\ensuremath{\mathcal{B}}\xspace}

\begin{enumerate}
	\item \(\braid\) for the greedy braid of the \(\frac{n}{2} \times m\) table;
	\item \(\VD\) for the Voronoi diagram of \(\braid\) % the \(\frac{n}{2} \times m\) table;
	\item \(\braid^*\) for the upward \(\frac{n}{2} \times m\) greedy braid;
	\item \(\braid^{-h}\) for the \(\lr*{\frac{n}{2} + h} \times m\) greedy braid that starts \(h\) rows above the middle line;
	\item \(\VD^{-h}\) for the Voronoi diagram corresponding to \(\braid^{-h}\);
	\item \(s_0, \ldots, s_{m+n}\) for the sites of the Voronoi diagram;
	\item \(f_0, \ldots, f_{m+n}\); \(f_0^{-h}, \ldots, f_{m+n}^{-h}\) for the Voronoi cells of \(\VD\) and \(\VD^{-h}\) correspondingly;
	\item \(c_0, \ldots, c_{m+n}\); \(c_0^{-h}, \ldots, c_{m+n}^{-h}\) for the lower right corners of the Voronoi cells of \(\VD\) and \(\VD^{-h}\) correspondingly.
\end{enumerate}


\begin{figure}[ht] \centering
	\begin{tikzpicture}[scale=0.65]
	\draw
	\foreach \h / \t in {0 / A, 1 / A, 2 / C, 3 / C}
		{(-0.6,3.5-\h) node{{\sffamily\t}}}
	\foreach \w / \t in {0 / A, 1 / A, 2 / C, 3 / B, 4 / A, 5 / C}
		{(0.4+\w,4.5) node{{\sffamily\t}}};
 % ────── Границы областей Вороного
	\fill[c8,opacity=0.22] (0,0)\vr\vu\vl -- cycle;
	\fill[c9,opacity=0.22] (0,1)\vd\vr\vuu\vl -- cycle;
	\fill[ca,opacity=0.22] (0,2)\vd\vr\vu\vul -- cycle;
	\fill[cb,opacity=0.22] (0,3)\vd\vrd\vdd\vdd\vrr
		\vuu\vuu\vul\vlu\vul -- cycle;
	\fill[c1,opacity=0.22] (0,4)\vd\vrd\vdr\vrd\vdd\vdd\vrr\vrr
		\vu\vuu\vl\vlu\vu\vl\vlu\vul\vlu -- cycle;
	\fill[c2,opacity=0.22] (0.5,4)\vdr\vrd\vdr\vr\vuu\vl\vlu -- cycle;
	\fill[c3,opacity=0.22] (1.5,4)\vdr\vr\vu -- cycle;
	\fill[c4,opacity=0.22] (2.5,4)\vdd\vdd\vdr\vr\vuu\vuu\vu -- cycle;
	\fill[c5,opacity=0.22] (3.5,4)\vdd\vdd\vdd\vdd\vrr\vrr\vr
		\vuu\vu\vlu\vuu\vu\vl\vlu -- cycle;
	\fill[c6,opacity=0.22] (4.5,4)\vdr\vr\vu -- cycle;
	\fill[c7,opacity=0.22] (5.5,4)\vdd\vdd\vdr\vuu\vuu\vu -- cycle;
 % ────── Нити кос
	\draw[thick,green] (0,3.5)
	\hc\hc\hx\hx\hc\hx\hn
	\hc\hc\hx\hx\hc\hx\hn
	\hx\hx\hc\hx\hc\hc\hn
	\hx\hx\hc\hx\hx\hc;
 % ────── Диагональные рёбра
	\draw[thick,red]
	\foreach \x / \y in
		{0 / 3, 0 / 4, 1 / 3, 1 / 4,
		 2 / 1, 2 / 2, 4 / 3, 4 / 4,
		 5 / 1, 5 / 2}
		{(\x,\y) -- ++(1,-1)};
 % ────── Границы рисунка
	\draw[black,opacity=0.35] (0,0) grid (6,4);
	\draw (0,0) rectangle (6,4);
 % ────── Сайты и углы областей
	\draw
	\foreach \i / \ix / \iy in
		{1/3.5/0, 2/2.5/2.5, 3/2.5/3.5, 4/3.5/1.5,
		 5/6/0, 6/5.5/3.5, 7/6/1.5}
		{(-1+\i,4) node[circle,fill=c\i,inner sep=0.55mm]{ }
		 (\ix,\iy) \rdcor{\i}}
	\foreach \nom / \i / \ix / \iy in
		{0/8/0.5/0, 1/9/0.5/0.5, 2/a/0.5/1.5, 3/b/1.5/0}
		{(0,\nom) node[circle,fill=c\i,inner sep=0.55mm]{ }
		 (\ix,\iy) \rdcor{\i}};

 % ══════════════════

   \begin{scope}[xshift=8.5cm]
	\draw
	\foreach \h / \t in {0 / A, 1 / A, 2 / C, 3 / B,
	                     4 / A, 5 / A, 6 / C, 7 / C}
		{(-0.6,7.5-\h) node{{\sffamily\t}}}
	\foreach \w / \t in {0 / A, 1 / A, 2 / C, 3 / B, 4 / A, 5 / C}
		{(0.4+\w,8.5) node{{\sffamily\t}}};
 % ────── Границы областей Вороного
	\fill[c8,opacity=0.22] (0,0)\vr\vu\vl -- cycle;
	\fill[c9,opacity=0.22] (0,1)\vd\vr\vuu\vl -- cycle;
	\fill[ca,opacity=0.22] (0,2)\vd\vr\vu\vul -- cycle;
	\fill[cb,opacity=0.22] (0,3)\vd\vrd\vdd\vdd\vrr
		\vuu\vuu\vul\vlu\vul -- cycle;
	\fill[c1,opacity=0.22] (0,4)\vd\vrd\vdr\vrd\vdd\vdd\vrr\vrr
		\vuu\vul\vlu\vu\vl\vlu\vul\vlu -- cycle;
	\fill[c2,opacity=0.22] (0.5,4)\vdr\vrd\vdr\vr\vu\vul\vlu -- cycle;
	\fill[c3,opacity=0.22] (1.5,4)\vdr\vrd\vdd\vdr\vrd\vdd\vrr\vrr\vr
		\vu\vlu\vul\vlu\vul\vlu\vul\vlu -- cycle;
	\fill[c4,opacity=0.22] (2.5,4)\vdr\vrd\vdr\vrd\vdr\vrd\vdr
		\vuu\vlu\vu\vl\vlu\vul\vlu -- cycle;
	\fill[c5,opacity=0.22] (3.5,4)\vdr\vrd\vdr\vr\vu\vul\vlu -- cycle;
	\fill[c6,opacity=0.22] (4.5,4)\vdr\vrd\vdd\vdr\vuu\vuu\vlu -- cycle;
	\fill[c7,opacity=0.22] (5.5,4)\vdr\vu -- cycle;
 % ────── Нити кос
	\draw[thick,green] (0,5.5)
	\hx\hx\hc\hx\hx\hc\hn
	\hx\hx\hx\hc\hx\hc\hn
	\hc\hc\hc\hc\hc\hc\hn
	\hc\hc\hx\hc\hc\hx\hn
	\hx\hx\hc\hc\hc\hc\hn
	\hx\hx\hc\hx\hx\hc;
 % ────── Диагональные рёбра
	\draw[thick,red]
	\foreach \x / \y in
		{0 / 3, 0 / 4, 1 / 3, 1 / 4,
		 2 / 1, 2 / 2, 4 / 3, 4 / 4,
		 5 / 1, 5 / 2,
		 2 / 6, 3 / 5, 5 / 6}
		{(\x,\y) -- ++(1,-1)};
 % ────── Границы рисунка
	\draw[black,opacity=0.35] (0,0) grid (6,8);
	\draw (0,0) rectangle (6,8);
	\draw (0,6) -- (6,6);
 % ────── Сайты и углы областей
	\draw
	\foreach \i / \ix / \iy in
		{1/3.5/0, 2/2.5/2.5, 3/6/0, 4/6/0.5,
		 5/5.5/2.5, 6/6/1.5, 7/6/3.5}
		{(-1+\i,4) node[circle,fill=c\i,inner sep=0.55mm]{ }
		 (\ix,\iy) \rdcor{\i}}
	\foreach \nom / \i / \ix / \iy in
		{0/8/0.5/0, 1/9/0.5/0.5, 2/a/0.5/1.5, 3/b/1.5/0}
		{(0,\nom) node[circle,fill=c\i,inner sep=0.55mm]{ }
		 (\ix,\iy) \rdcor{\i}};
   \end{scope}

   \begin{scope}[xshift=3cm,yshift=6.5cm,rotate=45]
	\draw[<->,gray] (-1.8,0) -- (1.8,0);
	\draw (0.65,0.2) node[rotate=45,text depth=0.8ex]{{\footnotesize Right}}
	     (-0.75,0.2) node[rotate=45,text depth=0.8ex]{{\footnotesize Left}};
   \end{scope}

	\draw (3,-1) node{(a)} (11.5,-1) node{(b)};
\end{tikzpicture}


	\begin{tikzpicture}[scale=0.65]
	\draw
	\foreach \h / \t in {0 / A, 1 / A, 2 / C, 3 / C}
		{(-0.6,3.5-\h) node{{\sffamily\t}}}
	\foreach \w / \t in {0 / A, 1 / A, 2 / C, 3 / B, 4 / A, 5 / C}
		{(0.4+\w,4.5) node{{\sffamily\t}}};
 % ────── Границы областей Вороного
	\fill[c8,opacity=0.22] (0,0)\vr\vu\vl -- cycle;
	\fill[c9,opacity=0.22] (0,1)\vd\vr\vuu\vl -- cycle;
	\fill[ca,opacity=0.22] (0,2)\vd\vr\vu\vul -- cycle;
	\fill[cb,opacity=0.22] (0,3)\vd\vrd\vdd\vdd\vrr
		\vuu\vuu\vul\vlu\vul -- cycle;
	\fill[c1,opacity=0.22] (0,4)\vd\vrd\vdr\vrd\vdd\vdd\vrr\vrr
		\vu\vuu\vl\vlu\vu\vl\vlu\vul\vlu -- cycle;
	\fill[c2,opacity=0.22] (0.5,4)\vdr\vrd\vdr\vr\vuu\vl\vlu -- cycle;
	\fill[c3,opacity=0.22] (1.5,4)\vdr\vr\vu -- cycle;
	\fill[c4,opacity=0.22] (2.5,4)\vdd\vdd\vdr\vr\vuu\vuu\vu -- cycle;
	\fill[c5,opacity=0.22] (3.5,4)\vdd\vdd\vdd\vdd\vrr\vrr\vr
		\vuu\vu\vlu\vuu\vu\vl\vlu -- cycle;
	\fill[c6,opacity=0.22] (4.5,4)\vdr\vr\vu -- cycle;
	\fill[c7,opacity=0.22] (5.5,4)\vdd\vdd\vdr\vuu\vuu\vu -- cycle;
 % ────── Нити кос
	\draw[thick,green] (0,3.5)
	\hc\hc\hx\hx\hc\hx\hn
	\hc\hc\hx\hx\hc\hx\hn
	\hx\hx\hc\hx\hc\hc\hn
	\hx\hx\hc\hx\hx\hc;
 % ────── Диагональные рёбра
	\draw[thick,red]
	\foreach \x / \y in
		{0 / 3, 0 / 4, 1 / 3, 1 / 4,
		 2 / 1, 2 / 2, 4 / 3, 4 / 4,
		 5 / 1, 5 / 2}
		{(\x,\y) -- ++(1,-1)};
 % ────── Границы рисунка
	\draw[black,opacity=0.35] (0,0) grid (6,4);
	\draw (0,0) rectangle (6,4);
 % ────── Путь, не пересекающий нити дважды
	\draw[thick,DarkOrchid] (2,4) -- (2,2) -- (3,1) -- (5,1) -- (6,0);
	\draw[thick,DarkOrchid] (3,4) -- (6,1) -- (6,0.5);
	\fill[black] (4,1) circle[radius=1mm] node[anchor = north east]{\small \(p_1\)};
	\fill[black] (4,3) circle[radius=1mm] node[anchor = north east]{\small \(p_2\)};
 % ────── Сайты и углы областей
	\draw
	\foreach \i / \ix / \iy in
		{1/3.5/0, 2/2.5/2.5, 3/6/0, 4/6/0.5,
		 5/5.5/2.5, 6/6/1.5, 7/6/3.5}
		{(-1+\i,4) node[circle,fill=c\i,inner sep=0.55mm]{ }
		 (\ix,\iy) \rdcor{\i}}
	\foreach \nom / \i / \ix / \iy in
		{0/8/0.5/0, 1/9/0.5/0.5, 2/a/0.5/1.5, 3/b/1.5/0}
		{(0,\nom) node[circle,fill=c\i,inner sep=0.55mm]{ }
		 (\ix,\iy) \rdcor{\i}};

 % ══════════════════

	\draw (3,-1) node{(c)};
\end{tikzpicture}


	\caption{(a) Voronoi diagram for a given greedy braid, (b) \(\VD^{-2}\), (c) paths from \(s_6\) to \(c_6^{-2}\) through \(p_1\) and from \(s_7\) to \(c_7^{-2}\) through \(p_2\)}
	\label{fig:vd-ex}
\end{figure}


\section{Query}

\begin{problem}
	Given braid \braid, the lower right corners \(c_0^{-h}, \ldots, c_{m+n}^{-h}\) of the Voronoi cells of \(\VD^{-h}\), point \(p\) and a number \(i\), check whether \(p \in f_i^{-h}\), \(p\) is to the left or to the right from \(f_i^{-h}\).
\end{problem}

\begin{lemma}
	The following are equivalent:
   \begin{enumerate}
	\item Point \(p\) belongs to the Voronoi cell \(f_i^{-h}\) of \(\VD^{-h}\);
	\item There is a path from \(s_i\) to \(c_i^{-h}\) passing through \(p\) that crosses no strand of \braid twice.
   \end{enumerate}
\end{lemma}

The path can utilize diagonal edges, see Figure~\ref{fig:vd-ex},~(c): \(p_1 \in f_6^{-2}\) (green), \(p_2 \in f_7^{-2}\) (yellow).

\begin{figure}[ht] \centering
	\begin{tabular}{ccc}
\makecell[c]{\begin{tikzpicture}[xscale=0.6,yscale=-0.6]
	\fill[cb,opacity=0.25] (0,0) rectangle (6,4);
	\draw (0,0) rectangle (6,4)
	      (0,2.25) -- (6,2.25) (0,3.5) -- (6,3.5)
	      (2,0) -- (4.25,3.5);
	\draw[thick,green] (3.75,0)
	      .. controls (3.75,1) and (2.25,1.5) ..
	      (2.25,2.25)
	      .. controls (2.25,3) and (5.25,2.75) ..
	      (5.25,3.5);
	\fill[black]
	 (2,0) circle[fill=black,radius=1mm]
	       node[anchor = south west]{\small \(s_i\)}
	 (3.4464,2.25) circle[radius=1mm]
	               node[anchor = south west]{\small \(p\)}
	 (4.25,3.5) circle[radius=1mm]
	            node[anchor = south west]{\small \(c_i^{-h}\)};
\end{tikzpicture}} & \hspace{0.8cm} &
\makecell[c]{\begin{tikzpicture}[xscale=0.45,yscale=0.5]
 % ────── Задняя часть
	\draw[dashed,thick,->,IndianRed] \proj000 -- \proj070;
	\draw[dashed] \proj050 -- \proj000 \proj050 -- \proj054
	              \proj050 -- \proj650
	              \proj23{2.5} -- \proj234 \proj23{2.5} -- \proj63{2.5}
	              \proj23{2.5} -- \proj20{2.5};
 % ────── Отсечения сзади
	\draw[thick,HotPink,dotted]
	      \proj254 -- \proj250 -- \proj200
	      \proj65{2.5} -- \proj05{2.5} -- \proj00{2.5}
	      \proj630 -- \proj030 -- \proj034;
 % ────── Поверхность парал-пипеда
	\filldraw[draw=black,fill=cb,fill opacity=0.3]
	   \proj000 -- \proj600 -- \proj650 --
	   \proj654 -- \proj054 -- \proj004 -- cycle;
	\fill[c3,opacity=0.7] \proj20{2.5} -- \proj60{2.5} --
	                      \proj63{2.5} -- \proj63{2.5} --
	                      \proj634 -- \proj234 -- \proj204 -- cycle;
	\draw \proj604 -- \proj004 \proj604 -- \proj654 \proj604 -- \proj600;
 % ────── Координатные оси
	\draw[thick,->,IndianRed] \proj000 -- \proj800;
	\draw[thick,->,IndianRed] \proj000 -- \proj00{5.5};
 % ────── Отсечения на поверхности
	\draw[thick,HotPink] \proj254 -- \proj204 -- \proj200
	      node[anchor=north east]{\color{black}\small \(s_i\)}
	      \proj65{2.5} -- \proj60{2.5} -- \proj00{2.5}
	      node[anchor=east]{\color{black}\small \(c_i^{-h}\)}
	      \proj630 -- \proj634 -- \proj034
	      node[anchor=south east]{\color{black}\small \(p\)};
\end{tikzpicture}} \\
	(a) & & (b) \\
\end{tabular}


	\caption{(a) A strand crossing the path twice, (b) a parallelepiped query that finds this strand}
	\label{fig:paral-trans}
\end{figure}


%\bibliography{}{}
%\bibliographystyle{plain}

\end{document}
