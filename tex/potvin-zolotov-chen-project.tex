\documentclass[a4paper,11pt]{article}
\usepackage{subfiles}
\usepackage{f420}

\title{Geodesics, Shortest Paths, and Non-Isomorphic Nets}
\author{{\scshape Potvin} Nicolas \and {\scshape Zolotov} Boris}
\date{ULB,\quad November 2020}

\begin{document} \maketitle

\section{Bounds on the number of egde-to-edge gluings of squares}

In this section, we prove that the number of edge-to-edge gluings of $n$ squares is polynomial in $n$.

\begin{theorem}
	There is at most $(2n)^{36}$ edge-to-edge gluings of $n$ squares that satisfy Alexandrov's conditions.
\end{theorem}

\begin{proof}
Since we're considering gluings that satisfy Alexandrov's conditions, there is a polyhedron corresponding to each of them. Consider this polyhedron and triangulate it. Since this polyhedron is glued from squares, it is possible to draw each of its faces on square grid, the vertices of the face being also the vertices of the grid.

What is more, an edge shared by two faces must look the same on the drawings of these faces; an example can be seen in Figure~\ref{fig:edgesMeeting}. That means, it must have the same length and the same or differing by $\frac{\pi}{2}$ tilt angle. 

\begin{figure}[h] \centering
\tikz[scale=0.45]{
	\foreach \i in {-1,...,6} {
		\draw[gray] (-1,\i) -- (6,\i) (\i,-1) -- (\i,6);
	}
	\draw[very thick,red] (4,5) -- (5,2);
	\draw[thick] (5,2) -- ++(-2,-2) -- ++(-3,0) --
		++(0,2) -- ++(1,3) -- ++(3,0);
}\hspace{1.2cm}
\tikz[scale=0.45]{
	\foreach \i in {-1,...,6} {
		\draw[gray] (-1,\i) -- (6,\i) (\i,-1) -- (\i,6);
	}
	\draw[very thick,red] (0,4) -- (3,5);
	\draw[thick] (3,5) -- ++(2,-2) -- ++(0,-3) --
		++(-3,0) -- ++(-2,2) -- ++(0,2);
}
\caption{Highlighted edge looks the same on the drawings of two faces it belongs to}
\label{fig:edgesMeeting}
\end{figure}

The graph of the polyhedron is planar and has at most 8 vertices. Due to Euler's formula, it can have at most 18 edges.
\end{proof}

% \bibliography{boris-bac}{}
% \bibliographystyle{plain}

\end{document}
