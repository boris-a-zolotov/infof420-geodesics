\documentclass[a4paper,11pt]{article}
\usepackage{subfiles}
\usepackage{f420}

\title{Geodesics, Shortest Paths, and Non-Isomorphic Nets}
\author{{\scshape Potvin} Nicolas \and {\scshape Zolotov} Boris}
\date{ULB,\quad December 2020}

\begin{document} \maketitle

\section{Introduction}

Given a collection of 2D polygons, a \emph{gluing} or a \emph{net} describes a closed surface by specifying how to glue (a part of) each edge of these polygons onto (a part of) another edge. Alexandrov's uniqueness theorem~\cite{alex} states that any valid gluing that is homeomorphic to a sphere and that does not yield a total facial angle greater than $2\pi$ at any point, corresponds to the surface of a unique convex 3D polyhedron (doubly covered convex polygons are also regarded as polyhedra). Note that the original polygonal pieces might need to be folded to obtain this 3D surface.

Unfortunately, the proof of Alexandrov's theorem is highly non-constructive. The only known approximation algorithm to find the vertices of this polyhedron~\cite{kpd09-approx} has a (pseudopolynomial) running time really large in $n$, where $n$ is the total complexity of the gluing.

There is no known exact algorithm for reconstructing the 3D polyhedron, and in fact the coordinates of the vertices of the polyhedron might not even be expressible as a closed formula~\cite{bannister2014galois}.

Enumerating all possible valid gluings is also not an easy task, as the number of gluings can be exponential even for a single polygon~\cite{DDLO02}. However one valid gluing can be found in polynomial time using dynamic programming~\cite{DO07,lo96-dynprog}. Complete enumerations of gluings and the resulting polyhedra are only known for very specific cases such as the Latin cross~\cite{ddlop99} and a single regular convex polygon~\cite{DO07}.

The special case when the polygons to be glued together are all identical regular $k$-gons, and the gluing is \emph{edge-to-edge} was studied recently for $k \ge 6$~\cite{kl17-hex} and $k=5$~\cite{alz-penta}. The aim of this project is to study the case of $k=4$: namely, to {\it enumerate} all valid gluings of squares and {\it classify} them up to isomorphism.

\section{Methods}

\subsection{Interpretating and listing valid gluings}

\subsection{Chen—Han algorithm}

Chen—Han algorithm is presented in~\cite{chen-han}. Estimation for the running time is proved in~\cite{z-bachthesis}.

\section{Bounds on the number of egde-to-edge gluings of squares}

In this section, we prove that the number of edge-to-edge gluings of $n$ squares is polynomial in $n$.

\begin{theorem}
	There are $O \left( n^{36} \right)$ edge-to-edge gluings of at most $n$ squares that satisfy Alexandrov's conditions.
\end{theorem}

\begin{proof}
Since we're considering gluings that satisfy Alexandrov's conditions, there is a polyhedron corresponding to each of them. Consider this polyhedron and triangulate it. Since this polyhedron is glued from squares, it is possible to draw each of its faces on square grid, the vertices of the face being also the vertices of the grid.

What is more, an edge shared by two faces must look the same on the drawings of these faces; an example can be seen in Figure~\ref{fig:edgesMeeting}. That means, it must have the same length and the tilt angle that is the same or differing by $\frac{\pi}{2}$. 

\begin{figure}[h] \centering
\tikz[scale=0.45]{
	\foreach \i in {-1,...,6} {
		\draw[gray] (-1,\i) -- (6,\i) (\i,-1) -- (\i,6);
	}
	\draw[very thick,red] (4,5) -- (5,2);
	\draw[thick] (5,2) -- ++(-2,-2) -- ++(-3,0) --
		++(0,2) -- ++(1,3) -- ++(3,0);
}\hspace{1.2cm}
\tikz[scale=0.45]{
	\foreach \i in {-1,...,6} {
		\draw[gray] (-1,\i) -- (6,\i) (\i,-1) -- (\i,6);
	}
	\draw[very thick,red] (0,4) -- (3,5);
	\draw[thick] (3,5) -- ++(2,-2) -- ++(0,-3) --
		++(-3,0) -- ++(-2,2) -- ++(0,2);
}
\caption{Highlighted edge looks the same on the drawings of two faces it belongs to}
\label{fig:edgesMeeting}
\end{figure}

The graph of the polyhedron is planar and has at most 8 vertices. Due to Euler's formula, it can have at most 18 edges. For each edge, we can choose its projections on $x$ and $y$ axes once it is drawn on the grid. Since the edge is a part of a flat face, all the squares that intersect the edge are distinct. There is at most $n$ of them, which yields that both projections are at most $n$, so there is at most $n^2$ ways to choose the edge.

Once the projections of the edges are known, let us draw the faces on the grid. At every vertex, there is at most two ways to place the next edge, such that the convexity of the face is preserved, those differ by $\frac{\pi}{2}$, see Figure~\ref{fig:twoWays} for an example.

\begin{figure}[h] \centering
\tikz[scale=0.45]{
	\foreach \i in {0,...,6} {
		\draw[gray] (0,\i) -- (6,\i) (\i,0) -- (\i,6);
	}
	\draw[thick] (1,0)--(2,3);
	\draw[thick,red,dashed] (4,6)--(2,3)--(5,2);
}
\caption{There are two ways to place each edge preserving convexity of the face}
\label{fig:twoWays}
\end{figure}

This adds at most $2^{2 \cdot 18}$ ways to draw the net once the edges are known, which gives the total of at most $(2n)^{36}$ gluings.
\end{proof}

\begin{theorem}
	There are $\Omega \left( n^{\frac52} \right)$ edge-to-edge gluings of at most $n$ squares that satisfy Alexandrov's conditions.
\end{theorem}

\begin{proof}
We will construct a family of $\Omega \left( n^{\frac52} \right)$ distinct polyhedra that can be glued from squares edge-to-edge. Those will be doubly-covered convex polygons whose edges are either parallel to the axes or inclined by $\frac{\pi}{4}$, and vertices are vertices of the grid.

Consider a square $\left \lfloor \frac{\sqrt n}{2} \right \rfloor \times \left \lfloor \frac{\sqrt n}{2} \right \rfloor$ drawn on the square grid. If a polygon is drawn inside this square, and it satisfies the conditions of the previous paragraph, then doubly covered version of this polygon can be glued from squares and consists of at most $n$ squares.

The polygons will be constructed the following way: we will take a rectangle and cut it angles off. More formally, we will pick two points on either of short sides of the rectangle and shoot lines inclined by $\frac{\pi}{4}$ from them. The points can correspond, in this case there is one vertex of the polyhedron on the corresponding side of the rectangle. An example can be seen in Figure~\ref{fig:cutExA}: the points we picked are marked, and the areas that are cut off are highlighted.

\begin{figure}[h] \centering
\begin{subfigure}[t]{3.2cm} \centering
\tikz[scale=0.45]{ % (3,-0.7) node[text height=3ex]{(a)}
	\baserect
	\filldraw[fill=bluetri,thick,fill opacity=0.55] (6,0) -- (4,0) -- (6,2) -- (3,5) -- (6,5) -- cycle;
	\filldraw[fill=bluetri,thick,fill opacity=0.55] (0,0) -- (1,0) -- (0,1) -- cycle;
	\filldraw[fill=bluetri,thick,fill opacity=0.55] (0,3) -- (2,5) -- (0,5) -- cycle;
	\fill (6,2) circle[radius = 1.6mm]
		(0,3) circle[radius = 1.6mm]
		(0,1) circle[radius = 1.6mm];
} \caption{} \label{fig:cutExA} \end{subfigure} \hspace{0.6cm}
\begin{subfigure}[t]{3.2cm} \centering
\tikz[scale=0.45]{
	\baserect
	\filldraw[fill=bluetri,thick,fill opacity=0.55] (0,3) -- (0,5) -- (6,5) -- (6,1) -- (2,5) -- cycle;
	\draw[thick,red,dashed] (0,2) -- (3,5);
	\fill (6,1) circle[radius = 1.6mm]
		(0,2) circle[radius = 1.6mm]
		(0,3) circle[radius = 1.6mm]
		(0,0) circle[radius = 1.6mm]
		(6,0) circle[radius = 1.6mm];
} \caption{} \label{fig:cutExB} \end{subfigure} \hspace{0.6cm}
\begin{subfigure}[t]{3.2cm} \centering
\tikz[scale=0.45]{
	\baserect
	\filldraw[fill=bluetri,thick,fill opacity=0.55] (0,0) -- (4,4) -- (6,2) -- (6,5) -- (0,5) -- cycle;
	\draw[thick,dashed] (0,4) -- (6,4);
	\fill (0,0) circle[radius = 1.6mm]
		(6,0) circle[radius = 1.6mm]
		(6,2) circle[radius = 1.6mm];
} \caption{} \label{fig:cutExC} \end{subfigure}

\caption{(a) An example of a polygon produced by cutting angles of a rectangle. (b) Some ways to cut angles do not produce valid polygons. (c) A polygon can be obtained by cutting angles of several different rectangles.} \label{fig:cutEx}
\end{figure}


Define the length of the longer side of the rectangle by $a$, and length of the shorter side by $b$. There number of ways to choose a pair of points on each of shorter sides is
\[ \ll \frac{b(b+1)}{2} \rr^2. \]

However, some sets of points do not produce valid polygons, since the rays shot from them intersect not at a vertex of the grid. An example can be seen in Figure~\ref{fig:cutExB}: red dashed line intersects with the line from the right side in the middle of a cell. Still, at least half the pairs of upper points on the sides are valid: for each invalid pair, the pair where left point is raised by 1 is valid. We get that the number of ways to select a valid set of points is
\[ \left. \ll \frac{b(b+1)}{2} \rr^2 \right/ 4 . \]

Note that each polyhedron has at least one horizontal edge, since otherwise $a$ must be less than $b$: the sum of segments cut off horizontal edges and the sum of segments cut off vertical edges are equal. However, on the other side there may not be a horizontal edge, see Figure~\ref{fig:cutExC}. In this case, it is not certain from which rectangle the polygon is cut: in Figure~\ref{fig:cutExC} it can be either $6 \times 4$ or $6 \times 5$.

However, one polygon can be cut from at most $a$ rectangles, since we know its horizontal dimension, and at least one horizontal side is fixed. Thus, the number of rectangles we get for a given $a$ is at least
\[ \sum\limits_{b=1}^a \frac{b^2 (b+1)^2}{16a} = \Omega\ll a^4 \rr. \]

Recall that $a$ can vary from 1 to $\sqrt n / 2$, the number of polygons we can produce is
\[ \Omega \ll \sum\limits_{a=1}^{\sqrt n / 2} a^4 \rr = \Omega \ll n^\frac{5}{2} \rr. \]

\end{proof}

\section*{Appendix: examples of valid nets}

% \begin{figure}[h] \centering
  \tikz[scale=0.56]{
    \fill[white] (-0.4,-0.4) rectangle (3+0.4,3+0.4);
    \foreach \i in {0,...,3} {\draw[gray,opacity=0.6] (0,\i) -- (3,\i) (\i,0) -- (\i,3);}
    \draw[thick] (0,1) -- (1,0) \midnode{1} -- (2,0) \midnode{2} -- (2,1) \midnode{3} -- (1,2) \midnode{4} -- (0,2) \midnode{5} -- cycle \midnode{0};} \qquad
  \tikz[scale=0.56]{
    \fill[white] (-0.4,-0.4) rectangle (3+0.4,3+0.4);
    \foreach \i in {0,...,3} {\draw[gray,opacity=0.6] (0,\i) -- (3,\i) (\i,0) -- (\i,3);}
    \draw[thick] (2,1) -- (2,2) \midnode{0} -- (1,2) \midnode{5} -- (0,1) \midnode{4} -- (0,0) \midnode{3} -- (1,0) \midnode{2} -- cycle \midnode{1};} \qquad
\caption{Net 0} \end{figure}

\begin{figure}[h] \centering
  \tikz[scale=0.56]{
    \fill[white] (-0.4,-0.4) rectangle (3+0.4,3+0.4);
    \foreach \i in {0,...,3} {\draw[gray,opacity=0.6] (0,\i) -- (3,\i) (\i,0) -- (\i,3);}
    \draw[thick] (1,2) -- (0,1) \midnode{1} -- (0,0) \midnode{2} -- (1,0) \midnode{3} -- (2,1) \midnode{4} -- (2,2) \midnode{5} -- cycle \midnode{0};} \qquad
  \tikz[scale=0.56]{
    \fill[white] (-0.4,-0.4) rectangle (3+0.4,3+0.4);
    \foreach \i in {0,...,3} {\draw[gray,opacity=0.6] (0,\i) -- (3,\i) (\i,0) -- (\i,3);}
    \draw[thick] (2,1) -- (2,2) \midnode{0} -- (1,2) \midnode{5} -- (0,1) \midnode{4} -- (0,0) \midnode{3} -- (1,0) \midnode{2} -- cycle \midnode{1};} \qquad
\caption{Net 1879605} \end{figure}


% \begin{figure}[h] \centering
  \tikz[scale=0.56]{
    \fill[white] (-0.4,-0.4) rectangle (4+0.4,4+0.4);
    \foreach \i in {0,...,4} {\draw[gray,opacity=0.6] (0,\i) -- (4,\i) (\i,0) -- (\i,4);}
    \draw[thick] (0,3) -- (0,0) \midnode{0} -- (1,0) \midnode{1} -- (2,1) \midnode{2} -- cycle \midnode{3};} \qquad
  \tikz[scale=0.56]{
    \fill[white] (-0.4,-0.4) rectangle (4+0.4,4+0.4);
    \foreach \i in {0,...,4} {\draw[gray,opacity=0.6] (0,\i) -- (4,\i) (\i,0) -- (\i,4);}
    \draw[thick] (3,0) -- (1,2) \midnode{3} -- (0,1) \midnode{2} -- (0,0) \midnode{1} -- cycle \midnode{0};} \qquad
\caption{Net 49637490} \end{figure}

\begin{figure}[h] \centering
  \tikz[scale=0.56]{
    \fill[white] (-0.4,-0.4) rectangle (4+0.4,4+0.4);
    \foreach \i in {0,...,4} {\draw[gray,opacity=0.6] (0,\i) -- (4,\i) (\i,0) -- (\i,4);}
    \draw[thick] (1,1) -- (0,3) \midnode{0} -- (0,2) \midnode{1} -- (1,0) \midnode{2} -- cycle \midnode{3};} \qquad
  \tikz[scale=0.56]{
    \fill[white] (-0.4,-0.4) rectangle (4+0.4,4+0.4);
    \foreach \i in {0,...,4} {\draw[gray,opacity=0.6] (0,\i) -- (4,\i) (\i,0) -- (\i,4);}
    \draw[thick] (2,1) -- (2,2) \midnode{3} -- (0,1) \midnode{2} -- (0,0) \midnode{1} -- cycle \midnode{0};} \qquad
\caption{Net 60877856} \end{figure}

\begin{figure}[h] \centering
  \tikz[scale=0.56]{
    \fill[white] (-0.4,-0.4) rectangle (4+0.4,4+0.4);
    \foreach \i in {0,...,4} {\draw[gray,opacity=0.6] (0,\i) -- (4,\i) (\i,0) -- (\i,4);}
    \draw[thick] (1,0) -- (3,1) \midnode{0} -- (3,2) \midnode{1} -- (0,1) \midnode{2} -- cycle \midnode{3};} \qquad
  \tikz[scale=0.56]{
    \fill[white] (-0.4,-0.4) rectangle (4+0.4,4+0.4);
    \foreach \i in {0,...,4} {\draw[gray,opacity=0.6] (0,\i) -- (4,\i) (\i,0) -- (\i,4);}
    \draw[thick] (2,1) -- (3,2) \midnode{3} -- (0,1) \midnode{2} -- (0,0) \midnode{1} -- cycle \midnode{0};} \qquad
\caption{Net 70707676} \end{figure}

\begin{figure}[h] \centering
  \tikz[scale=0.56]{
    \fill[white] (-0.4,-0.4) rectangle (4+0.4,4+0.4);
    \foreach \i in {0,...,4} {\draw[gray,opacity=0.6] (0,\i) -- (4,\i) (\i,0) -- (\i,4);}
    \draw[thick] (0,4) -- (1,0) \midnode{0} -- (2,0) \midnode{1} -- (1,4) \midnode{2} -- cycle \midnode{3};} \qquad
  \tikz[scale=0.56]{
    \fill[white] (-0.4,-0.4) rectangle (4+0.4,4+0.4);
    \foreach \i in {0,...,4} {\draw[gray,opacity=0.6] (0,\i) -- (4,\i) (\i,0) -- (\i,4);}
    \draw[thick] (4,1) -- (4,2) \midnode{3} -- (0,1) \midnode{2} -- (0,0) \midnode{1} -- cycle \midnode{0};} \qquad
\caption{Net 81497520} \label{fig:notRect} \end{figure}

\begin{figure}[h] \centering
  \tikz[scale=0.56]{
    \fill[white] (-0.4,-0.4) rectangle (4+0.4,4+0.4);
    \foreach \i in {0,...,4} {\draw[gray,opacity=0.6] (0,\i) -- (4,\i) (\i,0) -- (\i,4);}
    \draw[thick] (0,1) -- (4,0) \midnode{0} -- (3,1) \midnode{1} -- (0,2) \midnode{2} -- cycle \midnode{3};} \qquad
  \tikz[scale=0.56]{
    \fill[white] (-0.4,-0.4) rectangle (4+0.4,4+0.4);
    \foreach \i in {0,...,4} {\draw[gray,opacity=0.6] (0,\i) -- (4,\i) (\i,0) -- (\i,4);}
    \draw[thick] (2,4) -- (1,4) \midnode{3} -- (0,1) \midnode{2} -- (1,0) \midnode{1} -- cycle \midnode{0};} \qquad
\caption{Net 103500700} \end{figure}

\begin{figure}[h] \centering
  \tikz[scale=0.56]{
    \fill[white] (-0.4,-0.4) rectangle (4+0.4,4+0.4);
    \foreach \i in {0,...,4} {\draw[gray,opacity=0.6] (0,\i) -- (4,\i) (\i,0) -- (\i,4);}
    \draw[thick] (2,3) -- (0,0) \midnode{0} -- (1,1) \midnode{1} -- (3,4) \midnode{2} -- cycle \midnode{3};} \qquad
  \tikz[scale=0.56]{
    \fill[white] (-0.4,-0.4) rectangle (4+0.4,4+0.4);
    \foreach \i in {0,...,4} {\draw[gray,opacity=0.6] (0,\i) -- (4,\i) (\i,0) -- (\i,4);}
    \draw[thick] (3,3) -- (2,4) \midnode{3} -- (0,1) \midnode{2} -- (1,0) \midnode{1} -- cycle \midnode{0};} \qquad
\caption{Net 111905612} \end{figure}

% \begin{figure}[h] \centering
  \tikz[scale=0.56]{
    \fill[white] (-0.4,-0.4) rectangle (4+0.4,4+0.4);
    \foreach \i in {0,...,4} {\draw[gray,opacity=0.6] (0,\i) -- (4,\i) (\i,0) -- (\i,4);}
    \draw[thick] (0,0) -- (2,0) \midnode{3} -- (2,2) \midnode{4} -- cycle \midnode{0};} \qquad
  \tikz[scale=0.56]{
    \fill[white] (-0.4,-0.4) rectangle (4+0.4,4+0.4);
    \foreach \i in {0,...,4} {\draw[gray,opacity=0.6] (0,\i) -- (4,\i) (\i,0) -- (\i,4);}
    \draw[thick] (0,0) -- (2,2) \midnode{0} -- (0,2) \midnode{1} -- cycle \midnode{2};} \qquad
  \tikz[scale=0.56]{
    \fill[white] (-0.4,-0.4) rectangle (4+0.4,4+0.4);
    \foreach \i in {0,...,4} {\draw[gray,opacity=0.6] (0,\i) -- (4,\i) (\i,0) -- (\i,4);}
    \draw[thick] (0,0) -- (2,0) \midnode{1} -- (1,1) \midnode{5} -- cycle \midnode{8};} \qquad
  \tikz[scale=0.56]{
    \fill[white] (-0.4,-0.4) rectangle (4+0.4,4+0.4);
    \foreach \i in {0,...,4} {\draw[gray,opacity=0.6] (0,\i) -- (4,\i) (\i,0) -- (\i,4);}
    \draw[thick] (1,0) -- (1,2) \midnode{4} -- (0,1) \midnode{6} -- cycle \midnode{5};} \qquad
  \tikz[scale=0.56]{
    \fill[white] (-0.4,-0.4) rectangle (4+0.4,4+0.4);
    \foreach \i in {0,...,4} {\draw[gray,opacity=0.6] (0,\i) -- (4,\i) (\i,0) -- (\i,4);}
    \draw[thick] (1,0) -- (2,1) \midnode{6} -- (0,1) \midnode{3} -- cycle \midnode{7};} \qquad
  \tikz[scale=0.56]{
    \fill[white] (-0.4,-0.4) rectangle (4+0.4,4+0.4);
    \foreach \i in {0,...,4} {\draw[gray,opacity=0.6] (0,\i) -- (4,\i) (\i,0) -- (\i,4);}
    \draw[thick] (0,0) -- (1,1) \midnode{8} -- (0,2) \midnode{7} -- cycle \midnode{2};} \qquad
\caption{Net 0} \end{figure}

\begin{figure}[h] \centering
  \tikz[scale=0.56]{
    \fill[white] (-0.4,-0.4) rectangle (4+0.4,4+0.4);
    \foreach \i in {0,...,4} {\draw[gray,opacity=0.6] (0,\i) -- (4,\i) (\i,0) -- (\i,4);}
    \draw[thick] (2,0) -- (2,2) \midnode{3} -- (0,2) \midnode{4} -- cycle \midnode{0};} \qquad
  \tikz[scale=0.56]{
    \fill[white] (-0.4,-0.4) rectangle (4+0.4,4+0.4);
    \foreach \i in {0,...,4} {\draw[gray,opacity=0.6] (0,\i) -- (4,\i) (\i,0) -- (\i,4);}
    \draw[thick] (0,2) -- (2,0) \midnode{0} -- (2,2) \midnode{1} -- cycle \midnode{2};} \qquad
  \tikz[scale=0.56]{
    \fill[white] (-0.4,-0.4) rectangle (4+0.4,4+0.4);
    \foreach \i in {0,...,4} {\draw[gray,opacity=0.6] (0,\i) -- (4,\i) (\i,0) -- (\i,4);}
    \draw[thick] (0,0) -- (2,0) \midnode{1} -- (1,1) \midnode{5} -- cycle \midnode{8};} \qquad
  \tikz[scale=0.56]{
    \fill[white] (-0.4,-0.4) rectangle (4+0.4,4+0.4);
    \foreach \i in {0,...,4} {\draw[gray,opacity=0.6] (0,\i) -- (4,\i) (\i,0) -- (\i,4);}
    \draw[thick] (1,0) -- (1,2) \midnode{4} -- (0,1) \midnode{6} -- cycle \midnode{5};} \qquad
  \tikz[scale=0.56]{
    \fill[white] (-0.4,-0.4) rectangle (4+0.4,4+0.4);
    \foreach \i in {0,...,4} {\draw[gray,opacity=0.6] (0,\i) -- (4,\i) (\i,0) -- (\i,4);}
    \draw[thick] (1,0) -- (2,1) \midnode{6} -- (0,1) \midnode{3} -- cycle \midnode{7};} \qquad
  \tikz[scale=0.56]{
    \fill[white] (-0.4,-0.4) rectangle (4+0.4,4+0.4);
    \foreach \i in {0,...,4} {\draw[gray,opacity=0.6] (0,\i) -- (4,\i) (\i,0) -- (\i,4);}
    \draw[thick] (0,0) -- (1,1) \midnode{8} -- (0,2) \midnode{7} -- cycle \midnode{2};} \qquad
\caption{Net 15780252} \end{figure}


\bibliography{boris-bac}{}
\bibliographystyle{plain}

\end{document}
