\documentclass[a4paper,11pt]{article}
\usepackage{yaklass,afterpage,fancyvrb}

\begin{document}

It all starts with implementing the algorithm of Chen \& Han that finds the shortest paths from a given point on the surface of a polyhedron given by its net to all the other points on the surface. The idea of the algorithm is to shoot a ray from the point in every possible direction and see where the rays can reach and which edges of the polyhedron these rays intersect.

The algorithm can be implemented for squares, that are glued edge-to-edge: such a net is easy to store and draw. A harder approach can be taken with a polyhedron being spanned by vertices and edges drawn in 3D by user somehow.

A polynomial procedure that lists all possible nets of squares up to a given size can be implemented in several ways: from DFS-like approach to drawing all possible polygons on a square grid and checking which combinations of them fit together.

Chen—Han algorithm can be run on the net to determine the length of the shortest paths between the vertices. If there is a suitable permutation of vertices of two nets such that the lengths of paths between the vertices differ by a constant, then the nets are isomorphic and they correspond to the same polyhedron.

The whole procedure helps us clusterize the nets and only present non-isomorphic ones.

The shooting technique of Chen \& Han can be also used for other polyhedra-related problems like drawing deodesics.

To conclude, the general plan is

\begin{enumerate}
	\item Chen—Han for nets of squares
	\item All nets of $n$ squares
	\item Chen—Han for those nets to clusterize them
	\item See which of possible $\le 8$-hedra can be glued from squares
\end{enumerate}

This is already enough work but other related directions can also be studied.

\end{document}